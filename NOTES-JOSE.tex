\documentclass[12pt,epsfig]{article}
%\documentclass[aps,prd,preprint,showpacs,floatfix,preprintnumbers,nofootinbib,superscriptaddress,showkeys]{revtex4}
\usepackage[utf8]{inputenc}
\usepackage{graphicx}
\usepackage{rotating}
\usepackage{graphics}
\usepackage{float}
\usepackage{color}
\usepackage{fancybox}
\usepackage{hhline}
\usepackage{dcolumn}
\usepackage{textcomp}
\usepackage{epsfig,graphics,graphicx}
\usepackage{amsfonts,amssymb,amsmath}
\usepackage{pifont}
\usepackage{bm}
\usepackage{longtable} 
\usepackage{appendix}
\usepackage{lscape}
\usepackage[mathscr]{euscript}
\usepackage{mathrsfs}
\usepackage{multirow}
\usepackage{rotating}
\usepackage{color} 
\usepackage{placeins}
\usepackage[usenames]{color}
\usepackage{epsfig}
\usepackage{amsfonts}
\usepackage{amssymb,amsmath}
\usepackage{latexsym}
\usepackage[dvipsnames]{xcolor}
\def\red{\color{red}}
\def\blue{\color{blue}}
\def\green{\color{green}}
\def\violet{\color{violet}}
\def\orange{\color{orange}}
%\newcommand\appendix{\par \setcounter{chapter}{0}% \setcounter{section}{0}% \gdef\@chapapp{\appendixname}%\gdef\thechapter{\@Alph\c@chapter}}
\newcommand{\fr}{\frac}
\newcommand{\beq}{\begin{equation}}
\newcommand{\eeq}{\end{equation}}
\newcommand{\bea}{\begin{eqnarray}}
\newcommand{\eea}{\end{eqnarray}}
\newcommand{\zero}{{(0)}}
\newcommand{\one}{{(1)}}
\newcommand{\eps}{\epsilon}
\newcommand{\ord}[1]{{\cal{O}}( #1 )}
\newcommand{\B}{{\bf B}}
\newcommand{\Bdag}{{\bf B^\dagger}}
%\usepackage{youngtab2}
\newcommand{\cgsu}[3]{ \langle #1 #2 \mid #3\rangle}
\newcommand{\water}[2]{ #1,$^2$,#2}
\setlength{\textwidth}{6.49truein}
\setlength{\oddsidemargin}{-0.0in}
\setlength{\evensidemargin}{-0.0in} 
\setlength{\topmargin}{-0.35in}
\setlength{\textheight}{8.4in}%
\usepackage{epsfig}
\begin{document}
%\pagestyle{empty}
\def \ba{\begin{eqnarray}}\def\ea{\end{eqnarray}}
\def\bc{\begin{center}}\def\ec{\end{center}}
\def\nn{\nonumber\\}\def\v{\vec}

\title{\Huge \bf $\gamma\gamma \to \pi^0\pi^0}
\author{Jose}
\date{\today}
\maketitle
\pagestyle{plain}

  \section{The amplitude $\gamma^{(*)}\gamma^{(*)} \to \pi^0\pi^0$}
   \begin{center}
{\includegraphics[scale=0.75]{FIG-1.pdf}
\end{center}
 
  
  The relevant tensor is:
  \beq
  V_{\mu\nu}\equiv=\langle p_1,p_2\mid T (J_\mu (x) J_\nu (y))\mid 0\rangle
  \eeq
  where $J_\mu$ is the EM current.
  Fourier transforming in $x$ and $y$ with momenta $k_1$ and $k_2$ respectively, we can write the most general form for $V_{\mu\nu}$ which respects all symmetries:
  \beq
  V_{\mu\nu}=\sum _{i=1}^5 A_i(s,t,u) T^i_{\mu\nu}  \eeq
  where $s,~t,~u$ are  Mandelstam invariants and the tensor basis which respects gauge invariance is:
  \bea
  T_{\mu\nu}^1&=&   k_{1\;\nu} \;  k_{2\;\mu}-g_{\mu\nu} k_1 \cdot k_2     \nonumber\\
  T_{\mu\nu}^2&=&   k_{1\;\mu}\;   k_{1\;\nu}-g_{\mu\nu} k_1^2+\frac{1}{k_2\cdot P}(      k_{2\;\mu}\; k_1^2- k_{1\;\mu}\; k_1\cdot k_2)    \nonumber\\
  T_{\mu\nu}^3&=&    k_{2\;\mu} \;  k_{2\;\nu}-g_{\mu\nu} k_2^2+\frac{1}{k_1\cdot P}(      k_{1\;\nu}\; k_2^2- k_{2\;\nu} \;k_1\cdot k_2)           \nonumber\\
  T_{\mu\nu}^4&=& P_\mu P_\nu-\frac{1}{k_1\cdot k_2}(       k_{2\;\mu} \;P_\nu k_1\cdot P+ k_{1\;\nu} \;P_\mu k_2\cdot P-g_{\mu\nu} k_1\cdot P k_2\cdot P)    \nonumber\\
  T_{\mu\nu}^5&=&  k_{1\;\mu} k_{2\;\nu}-         \frac{1}{k_1\cdot k_2}(k_1^2 k_{2\;\mu} \;  k_{2\;\nu}+k_2^2   k_{1\;\mu} \;  k_{1\;\nu}-g_{\mu\nu} k_1^2 k_2^2)
  \eea
  with $P=p_1-p_2$, we have:
  \bea
  k_1\cdot k_2&=&\frac{s}{2}-k_1^2-k_2^2\nonumber\\
  k_1\cdot P&=&\frac{1}{2}(u-t+p_1^2-p_2^2)\nonumber\\
   k_2\cdot P&=&-\frac{1}{2}(u-t+p_2^2-p_1^2)
  \eea
  In the case $p_1^2=p_2^2$, $k_1\cdot P=-k_2\cdot P=\frac 12 (u-t)$.
  
  Bose symmetry requires that:
  \bea
  T_{\mu\nu}(P,k_1,k_2)&=&T_{\mu\nu}(-P,k_1,k_2)\nonumber\\
  &=& T_{\nu\mu}(P,k_2,k_1)
  \eea
  which corresponds also to the exchange $u\leftrightarrow t$.  This then implies that:
  \bea
  A_2(s,t,u)&=&A_3(s,u,t)\nonumber\\
  A_i(s,t,u)&=&A_i(s,u,t)~~~i=1,4,5
  \eea
  \section{$\pi^0\pi^0$ photoproduction }
  
  \subsection{Kinematics in Lab frame}
     \begin{center}
{\includegraphics[scale=0.75]{FIG-2.pdf}
\end{center}
 
  Definitions:
  
  \bea
  \omega=|\vec k|\nonumber\\
  \vec p_\pm&=& \vec p_1\pm \vec p_2,~~~~{\sf{p}}_\pm=|\vec p_\pm|\nonumber\\
  \vec p_f&=& \vec k-\vec p_+,~~~E_f=\sqrt{\vec p_f^2+M^2}
  \eea
  Spherical coordinates: choose $\vec k$ in $z$ direction.
  \bea
  \vec p_\pm &=& \sf{p}_\pm (\sin \theta_\pm \cos \phi_\pm,\sin \theta_\pm \sin \phi_\pm, \cos\theta_\pm)\nonumber\\
  E_1^2&=&\frac 14 ( \sf{p}_+^2+\sf{p}_-^2+2 \sf{p}_+\;\sf{p}_-\; \cos \alpha)+M_\pi^2\nonumber\\
   E_2^2&=&\frac 14 (  \sf{p}_+^2+\sf{p}_-^2-2 \sf{p}_+\;\sf{p}_-\; \cos \alpha)+M_\pi^2\nonumber\\
   \cos\alpha&=&\cos \theta_+\;\cos\theta_-+\cos(\phi_+-\phi_-)\sin \theta_+\;\sin\theta_-\nonumber\\
   \vec p_f^{~2}&=& \sf{p}_+^2+\omega^2-2 \sf{p}_+\; \omega \cos \theta_+\nonumber\\
   E_f^2&=&    \vec p_f^{~2}+M^2
  \eea
  so that $E_1+E_2=\omega+M-E_f$ depends only on $\sf{p}_+$ and $\theta_+$.
  From the above we get:
  \beq
  E_1-E_2=\frac{ \sf{p}_+ \sf{p}_- \cos\alpha}{\omega+M-E_f}=\frac{ \sf{p}_+ \sf{p}_- \cos\alpha}{E_1+E_2}
  \eeq
  \subsection{Differential cross section}
  \bea
  d\sigma&=& \frac{1}{2(4\pi)^5}\; \frac{ \mid {\cal{M}}\mid^2  }{\omega M E_1 E_2 E_f} \delta(\omega +M-E_1-E_2-E_f)\;d^3 p_+ d^3 p_-\nonumber\\
&=&  \frac{1}{2(4\pi)^5}\frac{ \mid {\cal{M}}\mid^2  }{\omega M E_1 E_2 E_f} \delta(\omega +M-E_1-E_2-E_f)\;     { \sf{ p}}_+ ^2 { \sf{ p}}_-^2 d\cos\theta_+ d\cos\theta_- d\phi_+ d\phi_- d{\sf{p}}_+ d{\sf{p}}_-\nonumber
  \eea
using that ${ \sf{ p}}_+\;  {\sf{ p}}_-\; \cos\alpha= \vec{{ p}}_+\cdot  \vec{{ p}}_-=E_1^2-E_2^2$, we obtain:
\beq
 \delta(\omega+M-E_1-E_2-E_f) =4\frac{E_1 E_2 {\sf{p}_-}}{(E_1+E_2)\mid {\sf{p}}_-^2- (E_1-E_2)^2\mid} \delta({\sf{p}}_--{\sf{\bar p}}_-)
\eeq  
  where
  \bea
{ \sf{\bar p}}_-&=&\frac{(E_1+E_2) \sqrt{(E_1+E_2)^2-{\sf{p}_+}^2-4M_\pi^2)}}{  \sqrt{(E_1+E_2)^2-\sf{p}_+^2\cos^2\alpha}}\nonumber\\
&=& \frac{\sqrt{{\sf{p}_+}^2 +W_{\pi\pi}} \sqrt{W_{\pi\pi}-4 M_\pi^2}}{\sqrt{W_{\pi\pi}+{\sf{p}_+}^2 \sin^2 \alpha}}  \eea
  Here we defined the squared $\pi\pi$ invariant mass:
  \bea
  W_{\pi\pi}&=&(E_1+E_2)^2-{ \sf{ p}}_+ ^2=2(\omega^2+M^2+\omega M)-2\omega { \sf{ p}}_+\; \cos\theta_+-2(\omega+M)E_f\nonumber\\
 &=&\left(-\sqrt{M^2+{ \sf{ p}}_+^2-2 { \sf{ p}}_+ \omega  \cos \theta _++\omega ^2}+M+\omega \right)^2-{ \sf{ p}}_+^2  \eea
  The diff cross section then becomes:
  \bea
  d\sigma &=&\frac{2}{(4\pi)^5}\;\frac{ \mid {\cal{M}}\mid^2  }{\omega M   E_f (E_1+E_2) \mid { \sf{\bar  p}}_-^2-(E_1-E_2)^2\mid} \; { \sf{ p}}_+ ^2 { \sf{\bar  p}}_-^3 \;d\cos\theta_+ d\cos\theta_- d\phi_+ d\phi_- d{\sf{p}}_+\nonumber\\
  &=&\frac{2}{(4\pi)^5}\;\frac{ \mid {\cal{M}}\mid^2  }{\omega M   E_f  \mid (E_1+E_2)- \frac{\sf{ p}_+^2\cos^2\alpha}{E_1+E_2}\mid} \; { \sf{ p}}_+ ^2 { \sf{\bar  p}}_- \;d\cos\theta_+ d\cos\theta_- d\phi_+ d\phi_- d{\sf{p}}_+
  \eea
  where we can use:
  \bea
  E_1+E_2&=& \omega+M-E_f\nonumber\\
  (E_1-E_2)^2&=&(E_1+E_2)^2-4 E_1 E_2\nonumber\\
  E_1 E_2&=&\sqrt{M_\pi^4+\frac 12 M_\pi^2({ \sf{ p}}_+ ^2+{ \sf{ p}}_- ^2)+\frac 14 ({ \sf{ p}}_+ ^4 +{ \sf{ p}}_- ^4-{ \sf{ p}}_+ ^2{ \sf{ p}}_- ^2 \cos(2\alpha))}
  \eea
  It is convenient to express the cross section in terms of the invariant mass squared of the two pion system, where $W_{\pi\pi}>4 M_\pi^2$ and 
  \beq
  d{ \sf{ p}}_+ =\frac{E_f}{2({ \sf{ p}}_+ (\omega+M)-\omega (E_1+E_2)\cos\theta_+)}dW_{\pi\pi}
  \eeq
  
  One can then write Eq(10) as:
   \beq
{ \sf{\bar p}}_-=\frac{(E_1+E_2) \sqrt{W_{\pi\pi}-4M_\pi^2}}{  \sqrt{W_{\pi\pi}+\sf{p}_+^2\sin^2\alpha}}
  \eeq
  
  
  
   With some work one can replace everywhere ${ \sf{ p}}_+ $ in terms of $W_{\pi\pi}$ using Eq. (12). For this, at a given $\omega$ and $\theta_+$,  one needs that:
   \beq
   W_{\pi\pi}^2- 4 W_{\pi\pi}  (M(M+\omega)+\omega^2 \sin^2\theta_+)
   +4 M^2 \omega^2>0
   \eeq
 and one gets:
 {\footnotesize
 \beq
{ \sf{ p}}_+  =\frac{\omega  \cos
   \theta_+ (2 M \omega +W_{\pi\pi})\pm (M+\omega ) \sqrt{ -4 M^2 \left(W_{\pi\pi}-\omega ^2\right)-4 M W_{\pi\pi} \omega +2 W_{\pi\pi} \omega ^2 \cos 2 \theta_++W_{\pi\pi} \left(W_{\pi\pi}-2 \omega ^2\right)}}{2 (M+\omega )^2-2 \omega ^2 \cos ^2\theta_+}
 \eeq}
  
  The next step is to determine the physical  domain of integration in the angles and $W_{\pi\pi}$. This is being worked out still.
  
  Also, one should find which angular variables are the most convenient to use. This requires that we know in detail the scattering amplitude's angular dependencies in order to make the choice.
  
 \subsection{Forward limit}
 We will be interested in the limit of large ${ \sf{ p}}_+$, small $\theta_+$ and small to moderate $W_{\pi\pi}$, which implies also small $\alpha$.. This also implies that We also want the limit of large $\omega$.
  In that limit we have:
  \bea
    { \sf{\bar p}}_-&=&\frac{(E_1+E_2) \sqrt{W_{\pi\pi}-4M_\pi^2}}{  \sqrt{W_{\pi\pi}+{ \sf{\bar p}}_+^2 \sin^2\alpha}}\nonumber\\
{ \sf{\bar p}}_+&=&\frac{\omega(W_{\pi\pi}+2 M \omega)+(M+\omega)\sqrt{W_{\pi\pi}(W_{\pi\pi}-4 M \omega)+4 M^2(\omega^2-W_{\pi\pi})}}{2 M(M+2\omega)}\nonumber\\
&=&\omega- \frac{W_{\pi\pi}}{2 \omega
   } -\frac{W_{\pi\pi}^2}{8 M \omega^2}-\frac{W_{\pi\pi}^2 \left(2 M^2+W_{\pi\pi}\right)}{16 M^2 \omega ^3}+\cdots\nonumber\\
d{ \sf{\bar p}}_+&=&\left(\frac{1}{2
   \omega } +\frac{W_{\pi\pi}}{4 M \omega ^2}+ \frac{W_{\pi\pi} \left(\frac{3 W_{\pi\pi}}{M^2}+4\right)}{16 \omega ^3}\right) dW_{\pi\pi}\nonumber\\
E_f&=& \frac{W_{\pi\pi}^3}{16 M^2 \omega ^3}+\frac{W_{\pi\pi}^2}{8 M \omega ^2}+M\nonumber\\
E_1+E_2&=& \omega- \frac{W_{\pi\pi}^2}{8 M \omega ^2}-\frac{W_{\pi\pi}^3  }{16 M^2 \omega ^3}~~~~~~~~~~~~~~~~~~~
  \eea
  
  
  \section{Primakoff amplitude and cross section}
  
\begin{center}
{\includegraphics[scale=0.75]{FIG-3.pdf}
\end{center}
 
  The scattering amplitude is given by the general expression:
  
  \beq
  {\cal{M}}=\epsilon^\mu T_{\mu\nu}(k,q,p_-) \frac{1}{Q^2} J^\nu
  \eeq
$T_{\mu\nu}$ is the Compton tensor,   $Q^2=-q^2$, and the target's EM current  in the Lab frame we will neglect the spin of the target, and therefore we only care about the its charge:
\beq
J^\mu=g^{\mu 0} Z e F(Q^2); \text{ note that we still need to use  }  q_\nu J^\nu=0
\eeq
  where $ F(Q^2)$ is the charge FF of the target.
  
Since we are interested in the region of the Primakoff  peak, first we approximate the amplitude by using the Compton tensor in the limit of real Compton scattering. This is then directly obtained from the result provided by Bellucci et al. which will be valid for the small $W_{\pi\pi}$ regime.  Later I will work out a more detailed analysis where the virtuality $Q^2$ is also included in the Compton tensor, and we will also need to give the amplitude for intermediate values of $W_{\pi\pi}$ (works of Oller and of Pennington).

For a scalar particle the (virtual)  Compton tensor is written in terms of five transverse tensors (see Bakker and Ji, Few-Body Syst
DOI 10.1007/s00601-016-1172-3)

\bea
T^{\mu\nu}&=&\sum_{n=1}^5 A_n(s,t,u) T_n^{\mu\nu}  \nonumber\\
T_1^{\mu\nu}&=& k\cdot q\; g^{\mu\nu}-k^\nu q^\mu   \nonumber\\
T_2^{\mu\nu}&=&(k^\mu k^\rho-k^2 \;g^{\mu\rho})(q_\rho q^\nu-q^2\;g_\rho^\nu)   \nonumber\\
T_3^{\mu\nu}&=& (p_-^\mu k^\rho-p_-\!\cdot k \;g^{\mu\rho})(p_-^\nu q_\rho-p_-\!\cdot q \;g_\rho^\nu)  \nonumber\\
T_4^{\mu\nu}&=& (p_-^\mu k^\rho-p_-\!\cdot k\; g^{\mu\rho})(q_\rho q^\nu-q^2\;g_\rho^\nu)+ (k^\mu k^\rho-k^2\; g^{\mu\rho})(p_-^\nu \,q_\rho-p_-\!\cdot q\; g_\rho^\nu)  \nonumber\\
T_5^{\mu\nu}&=& (k^\mu k^\rho-k^2 \;g^{\mu\rho})p_{-\rho}\; p_-^\sigma(q_\sigma q^\nu-q^2\;g_\sigma^\nu)  
\eea

Here, using $k^2=0$:
\bea
s&=& p_+^2=W_{\pi\pi}\nonumber\\
u-t&=&2 k \cdot p_-=2 \omega (E_1-E_2- { \sf{ p}}_- \cos \theta_-)\nonumber\\
s+t+u&=&2 M_\pi^2+q^2
\eea

which allow us to write:
\bea
q\cdot p_-&=& -k\cdot p_-=\frac{1}{2} (t-u)\nonumber\\
k\cdot q&=&\frac 12 (W_{\pi\pi}-q^2)\nonumber\\

\eea
In the limit $k^2=q^2=0$:
\bea
T_1^{\mu\nu}&=& k\cdot q\; g^{\mu\nu}-k^\nu q^\mu   \nonumber\\
T_2^{\mu\nu}&=&k\cdot q \; k^\mu  q^\nu   \nonumber\\
T_3^{\mu\nu}&=& (p_-^\mu k^\rho-p_-\!\cdot k \;g^{\mu\rho})(p_-^\nu q_\rho-p_-\!\cdot q \;g_\rho^\nu)  \nonumber\\
T_4^{\mu\nu}&=& (p_-^\mu k\cdot q-p_-\!\cdot k\; q^{\mu })  q^\nu+ k^\mu   (p_-^\nu \,k\cdot q-p_-\!\cdot q\; k^\nu)  \nonumber\\
T_5^{\mu\nu}&=&  k\cdot  p_- \; q\cdot  p_- \,k^\mu q^\nu 
\eea
where for real photons only $T_1$ and $T_3$ will contribute to the amplitude after contracting with photon polarizations. 
The tensor we need in this limit, eliminating terms that are proportional to $k^\mu$ due to transversity with the incoming photon polarization:
\bea
T_{\mu\nu}&=&A(W_{\pi\pi},t,u) (\frac 12 W_{\pi\pi} g_{\mu\nu}-k_\nu q_\mu)\\
&+&2 B(W_{\pi\pi},t,u)  ( (W_{\pi\pi}-q^2) p_{-\mu} p_{-\nu}-2(k\cdot p_-\; q_\mu p_{-\nu}+q\cdot p_{-} k_\nu p_{-\mu}-g_{\mu\nu} k \cdot p_-\; \;q\cdot p_-))\nonumber\\ 
&+& C(W_{\pi\pi},t,u) ( W_{\pi\pi}\,p_-^\mu -2\, p_-\! \cdot k\; q^\mu) q^\nu
\eea
where $p_-=p_1-p_2$. For the case where one contracts the Compton tensor with a conserved current as it is the case hare, namely $T^{\mu\nu} J_\nu$ where $J_\nu$ is the target electric current. Since $q^\mu J_\mu=0$, the term $C(W_{\pi\pi},t,u)$ does not contribute.

The low energy theorem for Compton scattering gives the following constraints:
\bea
\frac{\alpha}{2 M_\pi} (A+16 M_\pi^2 B)\arrowvert_{W_{\pi\pi}=0, t=M_\pi^2}&=& \alpha_{\pi}\nonumber\\
-\frac{\alpha}{2 M_\pi} A\arrowvert_{W_{\pi\pi}=0, t=M_\pi^2}&=& \beta_{\pi}
\eea
where $ \alpha_{\pi}$ $ \beta_{\pi}$ are the electric and magnetic polarizabilities respectively.

For the functions $A$ and  $B$ there are low energy results in ChPT (Bellucci et al) at two loops. The results are as follows:
\bea
A(s,t,u)&=& 4\frac{ G_\pi(s)}{s F_\pi^2}(s-M_\pi^2)+U_A+P_A\nonumber\\
 B(s,t,u)&=& U_B+P_B
\eea
where the functions and polynomials $U$ and $P$ are given in Bellucci's et al., see Appendix A:
\beq
G_\pi(s)=-\frac{1}{(4\pi)^2}\left (1+2\frac{M_\pi^2}{s}\int_0^1 \frac{dx}{x} \log(1-\frac{s}{M_\pi^2} x(1-x))\right)
\eeq
Use the integral in terms of dilogarithm functions:
\beq
\int_{0}^1 \frac{dx}{x} \log(1-U x(1-x))=-\text{Li}_2\left(\frac{1}{2} \left(U-\sqrt{U-4} \sqrt{U}\right)\right)-\text{Li}_2\left(\frac{1}{2} \left(U+\sqrt{U-4} \sqrt{U}\right)\right)\eeq
where in our case $U$ must be taken to have an imaginary part $+i\eps$.
At low energy $W_{\pi\pi}< (0.4 {\rm GeV})^2$ the $t$ dependence of the amplitudes $A$ and  $B$ is very small and can be neglected. We however should later consider also the effects of $Q^2>0$ and check that claim. 

\subsection{Amplitude squared in Lab frame}

 \bea
% \arrowvert {\cal{M}}\arrowvert^2&=&\frac{1}{Q^4} Z^2 e^2 F^2(Q^2)\; \arrowvert  A\, \omega\, \eps\cdot q - 2 B\;(E_1-E_2) ( (W_{\pi\pi}+Q^2+q\cdot p_- )\eps\cdot p_--2 k\cdot p_- \;\eps\cdot q)  \arrowvert^2\nonumber \\
 \arrowvert {\cal{M}}\arrowvert^2&=&\frac{1}{Q^4} Z^2 e^2 F^2(Q^2)\; \arrowvert  A(s,t,u)\, \omega\, \eps\cdot   q \nonumber\\
 &-& 2 B(s,t,u)(( (W_{\pi\pi}+Q^2) (E_1-E_2) +2\, \omega \, k\cdot p_- )  \eps\cdot  p_--2 (E_1-E_2) k\cdot p_- \; \eps\cdot   q)   \nonumber \\
 %&+& C(s,t,u) (E_1+E_2-\omega)(2 k\cdot p_- \;  \eps \cdot   q-W_{\pi\pi}  \eps\cdot   p_-)\arrowvert^2\nonumber\\
 &=&\frac{1}{Q^4} Z^2 e^2 F^2(Q^2)\; \arrowvert K_1\;  \eps \cdot   q+K_2\;   \eps \cdot   p_-\arrowvert^2
\eea

where
\bea
K_1&=&\omega A(s,t,u)+4 B(s,t,u) (E_1-E_2) k\cdot p_-\nonumber\\
K_2&=& -2 B(s,t,u)((W_{\pi\pi}+Q^2)(E_1-E_2) +2\omega k\cdot p_-)
\eea
The relevant products we need to use are:
\bea
\eps \cdot q&=&-\vec \eps \cdot \vec p_+=-{\sf{p}_+}\sin\theta_+ \cos\phi_+\nonumber\\
\eps\cdot p_-&=&-{\sf{p}_-}\sin\theta_- \cos\phi_-\nonumber\\
q^2&=&-Q^2=W_{\pi\pi}-2\omega(E_1+E_2-{\sf{p}_+}\ \cos\theta_+)\nonumber\\
k\cdot p_-&=& \omega(E_1-E_2-{\sf{p}_-}\cos\theta_-)\nonumber\\
q\cdot p_-&=&-k\cdot p_-= -\omega(E_1-E_2-{\sf{p}_-}\cos\theta_-)\nonumber\\
p_-^2&=&4 M_\pi^2-W_{\pi\pi}
\eea

In the case of unpolarized photon beam we get:
\bea
\arrowvert {\cal{M}}\arrowvert^2&=&\frac 12 \frac{1}{Q^4} Z^2 e^2 F^2(Q^2)\; (-|K_1|^2  p_-^2}+|K_2|^2 Q^2+2 Re(K_1 K_2^*)  \; p_-\!\cdot   k)
\eea


 \bea
 \arrowvert {\cal{M}}\arrowvert^2&=&\frac{1}{Q^4} Z^2 e^2 F^2(Q^2)\; ( A\, \omega\,   q^\mu - 2 B\;(E_1-E_2) ( (s+Q^2+q\cdot p_- ) p_-^\mu -2 k\cdot p_- \; q^\mu) \nonumber\\
 &\times& ( A^*\, \omega\,   q_\mu - 2 B^*\;(E_1-E_2) ( (s+Q^2+q\cdot p_- ) p_{-\mu} -2 k\cdot p_- \; q_\mu)\nonumber\\
 &=&\frac{e^2 Z^2 F\left(Q^2\right)^2}{Q^4} \biggl  (Q^2 \omega ^2 \Big(-\mid A\mid^2-16 \mid B\mid^2 ({E_1}-{E_2})^2   \nonumber\\ &\times&\left.
 (  ({E_1}+{E_2}) \cos ({\theta_-}) \sqrt{\frac{s-2 {M_\pi
   }^2}{{{ \sf{ p}}_+}^2 \sin ^2(\alpha )+s}}-{E_1}+{E_2})^2\Big )\nonumber\\
   &+&\left(\omega 
   ({E_1}+{E_2}) \cos ({\theta_-}) \sqrt{\frac{s-2 {M_\pi }^2}{{{ \sf{ p}}_+}^2
   \sin ^2(\alpha )+s}}-{E_1} \omega +{E_2} \omega +Q^2+s\right)\nonumber\\
   &\times&\left(4 Re(A B^*)  \omega ^2
   ({E_1}-{E_2}) \left(-({E_1}+{E_2}) \cos ({\theta_-}) \sqrt{\frac{s-2
   {M_\pi }^2}{{{ \sf{ p}}_+}^2 \sin ^2(\alpha )+s}}+{E_1}-{E_2}\right) \nonumber\\
   &+&16 \mid B\mid^2 \omega ^2
   ({E_1}-{E_2})^2 \left(({E_1}+{E_2}) \cos ({\theta_-}) \sqrt{\frac{s-2
   {M_\pi }^2}{{{ \sf{ p}}_+}^2 \sin ^2(\alpha )+s}}-{E_1}+{E_2}\right)^2\nonumber\\
   &+&4
   \mid B\mid^2 ({E_1}-{E_2})^2 \left(\frac{({E_1}+{E_2})^2 \left(2 {M_\pi
   }^2-s\right)}{{{ \sf{ p}}_+}^2 \sin ^2(\alpha )+s}+({E_1}-{E_2})^2\right) \nonumber\\
   &\times&\left.\left.\left(\omega 
   ({E_1}+{E_2}) \cos ({\theta_-}) \sqrt{\frac{s-2 {M_\pi }^2}{{{ \sf{ p}}_+}^2
   \sin ^2(\alpha )+s}}-{E_1} \omega +{E_2} \omega +Q^2+s\right)\right) \right)
\eea
\subsection{Amplitudes $A$ and $B$ for simulation}

We need to have a parametrization which for now gives a sufficiently realistic description for carrying out simulations. We present here a model for $A(s,t,u)$ and $B(s,t,u)$ based on dispersion theory to take into account the FSI of the pions with addition of t- and u- channel exchanges of resonances. At present there are many works   with increasing level of rigor, with the latest ones being quite complicated. So, for a first approximation we adopt here one of the early models  by Donoghue and Holstein (1994).

The Donoghue-Holstein model:  premises of the model: 1) only S-wave $\pi \pi$ FSI are considered, which is accurate at small $W_{\pi\pi}$. 2)  vector (also axial vector for the case of $\pi^+\pi^-$) exchanges in t- and u- channels of the $\gamma\gamma\to\pi\pi$ reaction. The latter give the t- and u- dependency.
For the neutral pions the model gives:
\bea
s\;A(s,t,u)&=& -\frac 23(f_0(s)-f_2(s))+\frac 23 (p_0(s)-p_2(s))-\frac s 2\sum_{V=\rho,\omega} R_V (\frac{t+M_\pi^2}{t-M_V^2}+\frac{u+M_\pi^2}{u-M_V^2})\nonumber\\
B(s,t,u)&=&-\frac 18 \sum_{V=\rho,\omega} R_V (\frac{1}{t-M_V^2}+\frac{1}{u-M_V^2})
\eea
Here the subindices 0 and 2 indicate the Isospin state of the pion pair. $f_I(s)$ admit a dispersive representation  for which we need the $I=0$ and 2 S-wave $\pi\pi$ phase shifts.
In the case of the neutral pions we only have $\rho_0$ and $\omega$ exchanges, where here $R_V$ indicates the coupling in the vertex $V\to \gamma \pi^0$.
Here we use:
\beq
R_V=\frac{6 M_V^2}{\alpha} \frac{\Gamma(V\to \pi \gamma)}{(M_V^2-M_\pi^2)^3}
\eeq
The functions $p_I(s)$ are known (given below by the model of resonance t- and u-channel exchanges), are real for positive $s$ and reproduce the smae discontinuity as $f_I(s)$ for negative $s$ . 

The hard problem is to implement the dispersive representation for $f_I(s)$. We will neglect inelasticities in the $\pi \pi$ FSI.
The DH model implements the approach of Morgan and Pennington who write a twice subtracted  dispersion relation for the combination $(f_I(s)-p_I(s))/\Omega_I(s)$, which has only a discontinuity for $s>4 M_\pi^2$ and is otherwise analytic everywhere. $\Omega_I(s)$ is the Omn\`es function given in terms of the corresponding $\pi\pi$ phase shift:
\bea
\Omega_I(s)&=&\exp\left(\frac s\pi\int_{4 M_\pi^2}^\infty \frac{\phi_I(s') }{s'-s} \frac{ds'}{s'}\right)\nonumber\\
\Omega_I(s>4 M_\pi^2)&=&e^{i\phi_I(s)} \exp\left(\frac s\pi\int_{4 M_\pi^2}^\infty \frac{\phi_I(s')-\phi_I(s)}{s'-s} \frac{ds'}{s'}+\frac{\phi_I(s)}{\pi}\log\frac{4M_\pi^2}{s-4M_\pi^2}\right)
\eea
where the phases $\phi_0$ is related to the corresponding $\pi\pi$ phase shift: $\phi_0(s)=\theta(M-\sqrt{s}) \delta_0^0(s)+\theta( \sqrt{s}-M)(\pi-\delta_0^0(s))$, where $M$ is the mass of the $f_0$ resonance. For $I=2$ one can take $\phi_2(s)=\delta_0^2(s)$.\\
We have that:
\bea
f_I(s)&=& p_I(s)+\Omega_I(s)\left(c_I+d_I\;s-\frac{s^2}{\pi} \int_{4M_\pi^2}^\infty p_I(s') \Im(\Omega_I^{-1}(s'))\frac {ds'} {(s'-s) s'^2}\right)
\eea


Finally, the functions $p_I(s)$ are as follows:

\bea
p_I(s)&=& f_I^{\rm Born}(s)+p_I^A(s)+p^\rho_I(s)+p_I^\omega(s)\nonumber\\
p_0^A(s)=p_2^A(s)&=&\frac{L_9^r+L_{10}^r}{F_\pi^2}\left(s+\frac{M_A^2-M_\pi^2}{\beta(s)}\log\frac{1+\beta(s)+s_A/s}{1-\beta(s)+s_A/s}\right)\nonumber\\
p_0^\rho(s)&=&\frac 32 R_\rho\left( \frac{M_\rho^2}{\beta(s)}  \log\frac{1+\beta(s)+s_\rho/s}{1-\beta(s)+s_\rho/s}\right)\nonumber\\
p_2^\rho(s)&=&0\nonumber\\
p_0^\omega(s)=-\frac 12 p_0^\omega(s)&=&-\frac 12 R_\omega\left( \frac{M_\omega^2}{\beta(s)}  \log\frac{1+\beta(s)+s_\omega/s}{1-\beta(s)+s_\omega/s}-s\right)
\eea
where $L_9^r$ and $L_{10}^r$ are the renormalized ${\ord{p^4}}$ LECs and given by: $L_9^r+L_{10}^r=1.43\pm 0.27\times 10^{-3}$, and:
\bea
s_i&=&2 (M_i^2-M_\pi^2)\nonumber\\
R_\omega&=& 1.35/GeV^2;~~~R_\rho=0.12/GeV^2\nonumber\\
\beta(s)&=&\sqrt{\frac{s-4 M_\pi^2}{s}}
\eea
  
  Numerical implementation: we need a simple functional parametrization of the phase shifts (in progress). 
  
  \subsection{Parametrization of the S-wave $\pi\pi$ phase shifts }
  
The following figures show the present knowledge of the relevant phase shifts:
\begin{center}
{\includegraphics[scale=0.52]{I=0-pi-pi-S-wave-Phase-shift.pdf}{\includegraphics[scale=0.5]{I=2-pi-pi-S-wave-Phase-shift.pdf}
\end{center}\\\\

\begin{center}
{\includegraphics[scale=0.79]{I=0-pi-pi-S-wave-Phase-shiftp.pdf}{\includegraphics[scale=0.5]{I=2-pi-pi-S-wave-Phase-shiftp.pdf}
\end{center}\\
 
 ~\\\\
  
  We fit to the phase shift data the following forms:
  
  \bea
  \delta_0^I(s)&=& \arcsin\left(\frac{\Gamma_I}{2\sqrt{(\sqrt{s} - M_I)^2 +\frac{ \Gamma_I^2}{4}}}\right)+\sum_{n=0}^N a_n (\sqrt{s})^n
  \eea
  where we include one single resonance for each $I=0,2$.
  
  For the available data we need only up to $N=3$ for $I=0$, with the result:
  \bea
  M_0=0.994 GeV;&&\Gamma_0=0.0624 GeV\nonumber\\
  a_0=-1.439;&&a_1=6.461 /GeV;~~a_2=-5.529 /GeV^2;~~a_3=2.022 /GeV^3
  \eea
For the case $I=2$ one finds that   the resonance term is not needed at all and a good fit is provided with $N=3$ with the result:
\beq
 a_0=-0.878;&&a_1=-0.611 /GeV;~~a_2=-0.083 /GeV^2;~~a_3=0.115 /GeV^3
\eeq

The figure shows the parametrized phase shifts along with the corresponding phase $\phi_I^0$.

\begin{center}
{\includegraphics[scale=0.62]{PhS-00.pdf}~~~~~{\includegraphics[scale=0.62]{PhS-02.pdf}}
\end{center}\\
 The resulting Omn\`es functions are shown in the figure, where the red curves show the real part and blue the imaginary ones.:
 
 \begin{center}
{\includegraphics[scale=0.62]{Omnes-00.pdf}~~~~~{\includegraphics[scale=0.62]{Omnes-02.pdf}}\\
\includegraphics[scale=0.82]{OmnesFunctionsFig.pdf}
\caption{We used the integration in $s'$ up to 1.4 and 2.4 $GeV^2$ to see how sensitive is the Omne\`es function to the high energy inputs of the phase shifts. In red the real parts and in blue the imaginary parts. }
\end{center}\\

The amplitude $A(s,t,u)$ in the limit of real photons, and in the approach followed here of keeping only the S-wave, becomes only a function of $s$. Using the results obtained here and and a fit to the available cross section (see the next section), one obtains that $A$ is real and given by the figure below. We are working on a parametrization at this point.
	\begin{figure}
\begin{center}

	{\includegraphics[scale=0.9729]{AmplA-Fig.png}}\\
\caption{Red: $ Re(A(s))$,~Blue: $Im((A(s))$}

		\end{center}
	\end{figure}\\

{\red Still working on improving parametrization of the Amplitude A}

Result of one parametrization: since we are using the S-wave approximation, there is no dependency on $t$, so:
\bea
Re\,A_{par}(s)&=&-\frac{0.0401449}{\left| s-(0.99\, -i\;0.027 )\right| }-3.42726\; s\nonumber\\
&+&0.00399002\; e^{5.90315 (s-1.48297)^2}-0.00202174\; e^{8.10357
	(s-1.37113)^2}+0.000320964 e^{9.36221 (s-1.32585)^2}
\nonumber\\
&+&0.000249196\; e^{9.41227 (s-1.31682)^2}+0.000158993 \;e^{8.04004
	(s-1.25463)^2}\nonumber\\
&+&531.611\; \tanh (21.5329 (s-0.00549155))-522.191
\eea
The imaginary part of $A$ is very small so we neglect it for now.


\section{$\gamma\gamma\to\pi^0\pi^0$ Cross section }

For real photons the  $\gamma\gamma\to\pi^0\pi^0$ cross section becomes:
\bea
\sigma_{\gamma\gamma\to\pi^0\pi^0}(|\cos \theta|<Z)(s)&=&\frac{\pi \alpha_{EM}^2}{s^2}
\int_{-Z}^Z  \frac 14  \sqrt{s (s -4 M_\pi^2)} \\
&\times&(\mid A(s,t,u) s-M_\pi^2  B(s,t,u)\mid^2+\frac {1}{s^2} (M_\pi^4- t\,u)^2\mid B(s,t,u)\mid ^2) dz\nonumber
\eea
where $z=\cos\theta$ and CM we have: $s+t+u=2 M_\pi^2$, and $t=M_\pi^2-\frac s 2+\frac 14\sqrt{s(s-4M_\pi^2)} z$.


Note that for both amplitudes $A$ and $B$ the dominant component for $s<1\;GeV^2$ is the S-wave, so that we have $A=A(s)$ and $B=B(s)$. Using that\\
 $t\;u=\frac{1}{16} \left(s \;z^2 \left(4 {M_\pi
	}^2-s\right)+4 \left(s-2 {M_\pi }^2\right)^2\right)$\\
where $\cos \theta=z$.

\bea
\sigma_{\gamma\gamma\to\pi^0\pi^0}(|\cos \theta|<Z)(s)&=&\frac{\pi \alpha_{EM}^2}{s^2}
    \frac Z2  \sqrt{s (s -4 M_\pi^2)} \\
&\times&(\mid A(s) s-M_\pi^2  B(s)\mid^2\nonumber\\
&+&\frac {1}{s^2} \left(M_\pi^4- \frac{1}{16}(\frac{Z^2}{3} s(4 M_\pi^2-s)+4(s-2 M_\pi^2)^2)\right)\mid B(s)\mid ^2) \nonumber
\eea



We fit to the Cristal Ball data the parameters $c_0,\;d_0,\;c_2,\;d_2$ which give in corresponding units: 
\bea
c_0&=& -0.529 \nonumber\\
d_0&=&-2.033 \nonumber\\
c_2&=&0.953\nonumber\\
d_2&=& -1.271\nonumber\\
\eea

The result is shown in the figure:\\\\
\begin{figure}
 \begin{center}
{\includegraphics[scale=0.92]{CrossSectionFig.pdf}
\caption{ $\gamma\gamma\to\pi^0\pi^0$ cross section integrated for the scattering angle $|\cos \theta|<0.8$. Data from the Cristal Ball}
\end{center}
\end{figure}

{\red In progress: a fit to a wider range in $s$ to include if possible the region of the $f_0$}

\section{Possible hadronic exchange background}

The possible hadronic t-exchange that can contribute to the $\pi^0\pi^0$ coherent photoproduction  will involve    $\rho^0$ and $\omega$ exchanges. We need to model this.

\section{Appendix A}
\subsection{$U_A$ and $P_A$ in ChPT (Bellucci et al)}
\bea
U_A&=&\frac{2}{s F_\pi^4}G_\pi(s)\left(  (s^2-M_\pi^2) J_\pi(s)+C(s)\right)+\frac{\ell_\Delta}{24 \pi^2 F_\pi^4}(s-M_\pi^2) J_\pi(s)\nonumber\\
&+&\left . \frac{\ell_2-5/6}{144 \pi^2 s F_\pi^4}(s-4 M_\pi^2)(H(s)+4(s G_\pi(s)+2 M_\pi^2(\tilde G_\pi(s)-3\tilde J_\pi(s)))d^2_{00})\nonumber\\
P_A&=&\frac{1}{(4\pi)^2 F_\pi^4}(a_1 M_\pi^2+a_2 s)
\eea
where the constants $a_1$ and $a_2$ need to be fitted, and:
\bea
J_\pi(s)&=&-\frac{1}{(4\pi)^2}\int_0^1 dx \log(1-\frac{s}{M_\pi^2}x(1-x))=\frac{2}{(4\pi)^2}\left(1-\frac{ \sqrt{4-\frac{s}{{M_\pi}^2}} \tan ^{-1}\left(\frac{\sqrt{\frac{s}{{M_\pi}^2}}}{\sqrt{4-\frac{s}{{M_\pi}^2}}}\right)}{\sqrt{\frac{s}{{M_\pi}^2}}}\right)\nonumber\\
\tilde{J_\pi}(s)&=& J_\pi(s)-s J'_\pi(0)\nonumber\\
\tilde{G_\pi}(s)&=& G_\pi(s)-s \,G'_\pi(0)\nonumber\\
H_\pi(s)&=&(s-10\, M_\pi^2)\;J_\pi(s)+6\,M_\pi^2\; G_\pi(s)
\eea
and:
\bea
C(s)&=&\frac{1}{48 \pi^2}\left(2(\ell_1-\frac 43)(s-2M_\pi^2)^2+\frac13(\ell_2-\frac 56)(4s^2-8sM_\pi^2+16 M_\pi^4)\right.\nonumber\\
&-&\left. 3 M_\pi^4 \ell_3+12 M_\pi^2(s-M_\pi^2)\ell_4-12 s M_\pi^2+15 M_\pi^4\right\Big)\nonumber\\
d^2_{00}&=&\frac 12(3 \cos^2 \theta_{CM}-1)
\eea
where  $\theta_{CM}$ is the $\gamma\gamma^*\to \pi\pi$ scattering angle in CM, and the low energy constants $\ell_i$ are known.

Note that the amplitude depends only on $s$ except for the term $d^2_{00}$. It is possible that this term will be entirely irrelevant at low $W_{\pi\pi}$ (need to check).
\subsection{$U_B$ and $P_B$}
\bea
U_B&=&\frac{\ell_2-\frac 56}{288\pi^2 F_\pi^4 s} H_\pi(s)\nonumber\\
P_B&=&\frac{b}{(4 \pi F_\pi)^4}
\eea
where $b$ is fitted.

\section{Appendix B}
CM kinematics


Useful invariants in Lab frame:
\bea
q&=& { \sf{ p}}_+-k\nonumber\\
\epsilon^\mu q_\mu&=& -\vec \epsilon \cdot \vec q=-\vec \epsilon \cdot  \vec  { \sf{ p}}_+=-{ \sf{ p}}_+\sin\theta_+\cos\phi_+\nonumber\\
\epsilon^\mu p_{-\mu}&=& -\vec \epsilon \cdot \vec { \sf{ p}}_-   =-{ \sf{ p}}_-\sin\theta_-\cos\phi_-\nonumber\\
k^\mu J_\mu&=& \omega Z e F(Q^2)\nonumber\\
k^\mu p_{+\mu}&=&k^\mu q_{\mu}=\omega(E_1+E_2-{ \sf{ p}}_+ \cos\theta_+)\nonumber\\
q^\mu p_{+\mu}&=&s-k^\mu p_{+\mu}\nonumber\\
Q^2&=&-s+2\, k^\mu p_{+\mu}=-s+2\omega((E_1+E_2)- { \sf{ p}}_+  \cos \theta_+)\nonumber\\
&=& 2 \omega { \sf{ p}}_+(1-\cos \theta_+)+s (\frac{\omega}{{ \sf{ p}}_+}-1)-s^2\frac{\omega}{4 { \sf{ p}}_+^3}+\cdots\nonumber\\
q^\mu p_{-\mu}&=&-k^\mu p_{-\mu}=-\omega(E_1-E_2-{ \sf{ p}}_- \cos\theta_-)
\eea


\bea
s&=& 4\; \omega_{CM} ^2=p_+ \cdot p_+\nonumber\\
-\frac 12 \sqrt{s(s-4M_\pi^2)}\cos \theta_{CM}&=& k\cdot p_-=\omega(E_1-E_2- { \sf{ p}}_-\cos\theta_-)
\eea
where we can use:
\bea
E_1+E_2&=& \sqrt{s+{ \sf{ p}}_+^2}\nonumber\\
E_1-E_2&=&\frac{{ \sf{ p}}_+ \sqrt{s-4 M_\pi^2}\cos\alpha}{\sqrt{s+{ \sf{ p}}_+^2\sin^2\alpha}}
\eeq
\newpage
\section{Relation between the amplitudes of Donoghue-Holstein and Bellucci-Gasser-Sainio}


The two basis Compton tensors are related by:

\bea
T^{DH}_{1\mu\nu}&=&-T^{BGS}_{1\mu\nu}\nonumber\\
T^{DH}_{2\mu\nu}&=&\frac 14 T^{BGS}_{1\mu\nu}+\frac 1{8s}  T^{BGS}_{2\mu\nu}
\eea

and the amplitudes are related as follows:
\bea
A^{DH}&=&-A^{BGS}-2 s B^{BGS}\nonumber\\
B^{DH}&=&8 s B^{BGS}
\eea

\section{ Primakoff Cross Section }

\bea
\frac{d^2\sigma}{d\theta\;d W_{\pi\pi}}&=&
(2 \pi sin\theta)\times 2 \alpha \frac{Z^2}{
\pi^2} E_\gamma^4\; \beta^2 \; \frac{sin^2\theta}{
W_{\pi\pi}}  \; \frac{\sigma(W_{\pi\pi})}{
Q^4} 
\eea
\begin{figure}
	\begin{center}
			{\includegraphics[scale=0.259]{XSection-3D.pdf}}\\
		\caption{Primakoff cross section for $\pi^0\pi^0$.}
		\end{center}
\end{figure}\\
\newpage
{\bf HOMEWORK }\\\\

1) Study the photon polarization dependency. (Dai \& Pennington studied the effect of polarization)\\

2) Another way to provide the amplitude $A$: spline or table.\\\\


 


\end{document}