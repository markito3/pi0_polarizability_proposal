\section{Introduction}

Electromagnetic polarizabilities are fundamental properties of
composite systems such as molecules, atoms, nuclei, and hadrons
\cite{Holstein:1990qy}. Whereas form factors provide information
about the ground state properties of a system, polarizabilities
provide information about the excited states of the system, and are
  therefore determined by the system's dynamics.
%For atomic systems polarizabilities are on the order of the atomic
%volume.  For hadrons the polarizabilities are much smaller than the
%volume, typically of the order of $10^{-4}$ fm$^3$, because of the
%greater stiffness of the QCD force as compared to the electromagnetic
%force.
Measurements of hadron polarizabilities provide an important test
point for Chiral Perturbation Theory, dispersion relation approaches,
and lattice calculations. Among the hadron polarizabilities, the
neutral pion polarizability is important because it
tests fundamental symmetries, in particular chiral symmetry and its
realization in QCD.  Indeed, the non-trivial (non-perturbative) vacuum
properties of QCD result in the phenomenon of spontaneous chiral
symmetry breaking, giving rise to the Goldstone boson nature of the
pions.  In particular, the Goldstone boson nature of the $\pi^0$
manifests itself most notably in its decay into $\gamma\gamma$ and
also in its electromagnetic polarizability, which according to ChPT
can be predicted to leading order in the expansion in quark
masses.


Hadron polarizabilities are best measured in Compton scattering
experiments where, in the case of nucleon polarizabilities, one looks
for a deviation of the cross section from the prediction of Compton
scattering from a structureless Dirac particle.
In the case of pions, the long lifetime of the charged pion permits
experiments of low energy Compton scattering using a beam of high
energy pions scattering on atomic electrons. On the other hand, the
short lifetime of the neutral pion requires an indirect study of low
energy Compton scattering via measurements of the process $\gamma
\gamma \rightarrow \pi^0 \pi^0$, a method that can also be applied to
the charged pion (CPP) and for which a proposal in Hall D is already
approved \cite{CPPexp}.

Measurements of hadron polarizabilities are among the most difficult
experiments performed in photo-nuclear physics. For charged hadrons,
because of the Born term, the polarizability effect in the cross
section can range from 10 to 20\% depending on the kinematics.  For
neutral hadrons, where the Born term is absent, the polarizability
effect will be much less than this.  To set reasonable expectations
for what can be accomplished in a measurement of this type, it is
important to recognize that after 30 years of dedicated experiments
using tagged photons at facilities across North America and Europe,
the error on the proton electric polarizability is 4\%, without doubt
the paramount experimental achievement in this field. However, the
error on the proton magnetic polarizability is 16\%
\cite{PDGTanabashi:2018oca}.  Absolute uncertainties provide a better
gauge of a measurements sensitivity; for proton electric and magnetic
polarizabilities the uncertainty in both is $\pm 0.4 \times
10^{-4}~\mathrm{fm}^3$.  Another level of precision to consider for
setting expectations is the result COMPASS obtained for charged pion
$\alpha - \beta$. COMPASS is also a Primakoff measurement. COMPASS
achieved a relative error of 46\% in $\alpha - \beta$ and an absolute
error of $\pm 0.9 \times 10^{-4}~\mathrm{fm}^3$.  COMPASS cannot
measure the neutral pion polarizability.



This proposal presents a plan to make a precision measurement of the
$\gamma \gamma^* \rightarrow \pi^0 \pi^0$ cross section using the
GlueX detector in Hall D.  The measurement is based on the Primakoff
effect which allows one to access the low $W_{\pi^0\pi^0}$ invariant
mass regime with the virtual photon $\gamma^*$ provided by the
Coulomb field of the target. The central aim of the measurement is to
drastically improve the determination of the cross section in this
domain, which is key for constraining the low energy Compton amplitude
of the $\pi^0$ and thus for extracting its polarizability.  At
present, the only accurate measurements exist for invariant masses of
the two $\pi^0$s above 0.7 GeV, far above the threshold 0.27 GeV. The
existing data at low energy were obtained in $e^+ e^- \to \pi^0\pi^0 $
scattering in the early 1990's with the Crystal Ball detector at the
DORIS-II storage ring at DESY \cite{Marsiske:1990hx}.

Meanwhile, theory has made significant progress over time, with
studies of higher chiral corrections 
\cite{Bellucci:1994eb,Gasser:2005ud,Aleksejevs:2014eea} and with the
implementation of dispersion theory analyses which serve to make use
of the higher energy data
\cite{Oller:2008kf,Dai:2014zta,Dai:2014lza,Moussallam:2013una}. It is
expected that the experimental data from this proposal, together with
these theoretical frameworks, will allow for the most accurate
extraction of the $\pi^0$ polarizabilities to date.
