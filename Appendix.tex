\documentclass[12pt,epsfig]{article}
%\documentclass[aps,prd,preprint,showpacs,floatfix,preprintnumbers,nofootinbib,superscriptaddress,showkeys]{revtex4}
\usepackage[utf8]{inputenc}
\usepackage{graphicx}
\usepackage{rotating}
\usepackage{graphics}
\usepackage{float}
\usepackage{color}
\usepackage{fancybox}
\usepackage{hhline}
\usepackage{dcolumn}
\usepackage{textcomp}
\usepackage{epsfig,graphics,graphicx}
\usepackage{amsfonts,amssymb,amsmath}
\usepackage{pifont}
\usepackage{bm}
\usepackage{longtable} 
\usepackage{appendix}
\usepackage{lscape}
\usepackage[mathscr]{euscript}
\usepackage{mathrsfs}
\usepackage{multirow}
\usepackage{rotating}
\usepackage{color} 
\usepackage{placeins}
\usepackage[usenames]{color}
\usepackage{epsfig}
\usepackage{amsfonts}
\usepackage{amssymb,amsmath}
\usepackage{latexsym}
\usepackage[dvipsnames]{xcolor}
\def\red{\color{red}}
\def\blue{\color{blue}}
\def\green{\color{green}}
\def\violet{\color{violet}}
\def\orange{\color{orange}}
%\newcommand\appendix{\par \setcounter{chapter}{0}% \setcounter{section}{0}% \gdef\@chapapp{\appendixname}%\gdef\thechapter{\@Alph\c@chapter}}
\newcommand{\fr}{\frac}
\newcommand{\beq}{\begin{equation}}
\newcommand{\eeq}{\end{equation}}
\newcommand{\bea}{\begin{eqnarray}}
\newcommand{\eea}{\end{eqnarray}}
\newcommand{\zero}{{(0)}}
\newcommand{\one}{{(1)}}
\newcommand{\eps}{\epsilon}
\newcommand{\ord}[1]{{\cal{O}}( #1 )}
\newcommand{\B}{{\bf B}}
\newcommand{\Bdag}{{\bf B^\dagger}}
%\usepackage{youngtab2}
\newcommand{\cgsu}[3]{ \langle #1 #2 \mid #3\rangle}
\newcommand{\water}[2]{ #1,$^2$,#2}
\setlength{\textwidth}{6.49truein}
\setlength{\oddsidemargin}{-0.0in}
\setlength{\evensidemargin}{-0.0in} 
\setlength{\topmargin}{-0.35in}
\setlength{\textheight}{8.4in}%
\usepackage{epsfig}
\begin{document}
%\pagestyle{empty}
\def \ba{\begin{eqnarray}}\def\ea{\end{eqnarray}}
\def\bc{\begin{center}}\def\ec{\end{center}}
\def\nn{\nonumber\\}\def\v{\vec}

\section{Theory appendix}
   The scattering amplitude for   $\gamma\gamma^*\to\pi^0\pi^0$ is given in terms of the Compton tensor, whose low energy expansion in the Compton scattering channel $\gamma\pi^0\to\gamma\pi^0$ is given in terms of the electric and magnetic polarizabilities of the $\pi^0$. For the case of interest with one real photon, the Compton tensor is given in by two amplitudes, namely:
   
   \bea
   T_{\mu\nu}&=&-(A(s,t,u)+\frac 14 B(s,t,u)) (\frac 12\, s \;g_{\mu\nu}-k_\nu q_\mu)\\
   &+&\frac{1}{4s} B(s,t,u)  ( (s-q^2) p_{-\mu} p_{-\nu}-2(k\cdot p_-\; q_\mu p_{-\nu}+q\cdot p_{-} k_\nu p_{-\mu}-g_{\mu\nu} k \cdot p_-\; \;q\cdot p_-)
   \eea
  Here $s=W_{\pi\pi}^2$ is the invariant mass squared of the two $\pi^0$s, $k$ the momentum of the beam photon, $q$ the momentum of the virtual photon, and $p_-$ the $p_-=p_1-p_2$ the momentum difference between the two pions.
  
  The limit of interest for the polarizabilities is:
  \bea
 \alpha_{\pi}&=& - \frac{\alpha}{2 M_\pi}  (A(s,t,u)-\frac 2 s M_\pi^2 B(s,t,u))\arrowvert_{s=0, t=u=M_\pi^2}\nonumber\\
 \beta_{\pi}&=&
  \frac{\alpha}{2 M_\pi} A\arrowvert_{s=0, t=u=M_\pi^2}
  \eea
  where $ \alpha_{\pi}$ $ \beta_{\pi}$ are the electric and magnetic polarizabilities respectively.
  
  The low energy limit is analyzed in ChPT. At the lowest significant order, i.e., one loop, the $\pi^0$ polarizabilities are entirely given in terms of known quantities, namely:
\bea
\alpha_{\pi_0}&=&-\beta_{\pi_0}= -\frac{\alpha}{96 \pi^2 M_\pi F_\pi^2}\simeq -0.55\times 10^{-4}~{\rm fm}^3
\eea
The positive magnetic susceptibility indicates that the $\pi_0$ is diamagnetic, and naturally the negative electric polarizability tells that it behaves as a dielectric.

There are higher order corrections in the chiral expansion to the above prediction   corresponding to a two-loop calculation, which is undefined up to two low energy constants $h_\pm$ in the notation of Ref. \cite{Bellucci:1994eb}, expected to be significant for the corrections. 

The amplitudes $A$ and $B$ are constrained by unitarity and analiticity to satisfy dispersion relations. In particular below $s\sim 0.8\text{ GeV}^2$ the dominant contributions are for the pair of pions in an S-wave. The rather well established S-wave phase shifts thus allow for implementing  dispersion relations \cite{Donoghue:1993kw, Oller:2008kf,Moussallam:2013una,Dai:2014zta,Dai:2014lza,Dai:2016ytz}.  In this proposal the model by Donoghue and Holstein \cite{Donoghue:1993kw} for implementing the dispersive representation using S-wave final state interaction is adopted. The model implements   twice subtracted dispersion relations for the isospin 0 and 2 components of the  amplitude A with the addition of t- and u-channel resonance exchanges for both A and B. The four subtraction constants require the experimental input of the cross section to be measured by the proposed experiment. 

A summary of useful  theory results is the following:  

1) representation of the Compton amplitudes:
\bea
s\;A(s,t,u)&=& -\frac 23(f_0(s)-f_2(s))+\frac 23 (p_0(s)-p_2(s))-\frac s 2\sum_{V=\rho,\omega} R_V (\frac{t+M_\pi^2}{t-M_V^2}+\frac{u+M_\pi^2}{u-M_V^2})\nonumber\\
B(s,t,u)&=&-\frac 18 \sum_{V=\rho,\omega} R_V (\frac{1}{t-M_V^2}+\frac{1}{u-M_V^2})\nonumber\\
R_V&=&\frac{6 M_V^2}{\alpha} \frac{\Gamma(V\to \pi \gamma)}{(M_V^2-M_\pi^2)^3}
\eea
where $V=\rho,\;\omega$, 
\bea
p_I(s)&=& f_I^{\rm Born}(s)+p_I^A(s)+p^\rho_I(s)+p_I^\omega(s)\nonumber\\
p_0^A(s)=p_2^A(s)&=&\frac{L_9^r+L_{10}^r}{F_\pi^2}\left(s+\frac{M_A^2-M_\pi^2}{\beta(s)}\log\frac{1+\beta(s)+s_A/s}{1-\beta(s)+s_A/s}\right)\nonumber\\
p_0^\rho(s)&=&\frac 32 R_\rho\left( \frac{M_\rho^2}{\beta(s)}  \log\frac{1+\beta(s)+s_\rho/s}{1-\beta(s)+s_\rho/s}\right)\nonumber\\
p_2^\rho(s)&=&0\nonumber\\
p_0^\omega(s)=-\frac 12 p_0^\omega(s)&=&-\frac 12 R_\omega\left( \frac{M_\omega^2}{\beta(s)}  \log\frac{1+\beta(s)+s_\omega/s}{1-\beta(s)+s_\omega/s}-s\right),
\eea
where $\beta(s)=\sqrt{\frac{s-4 M_\pi^2}{s}}$, $M_A$ the mass of the $A_1$ resonance. The $f_I$s are given by the dispersive representation:
\bea
f_I(s)&=& p_I(s)+\Omega_I(s)\left(c_I+d_I\;s-\frac{s^2}{\pi} \int_{4M_\pi^2}^\infty p_I(s') \text{Im}(\Omega_I^{-1}(s'))\frac {ds'} {(s'-s) s'^2}\right),
\eea
with the Omn\`es function:
\bea\Omega_I(s>4 M_\pi^2)&=&e^{i\phi_I(s)} \exp\left(\frac s\pi\int_{4 M_\pi^2}^\infty \frac{\phi_I(s')-\phi_I(s)}{s'-s} \frac{ds'}{s'}+\frac{\phi_I(s)}{\pi}\log\frac{4M_\pi^2}{s-4M_\pi^2}\right).
\eea
the phases $\phi_I$ are related to the corresponding $\pi\pi$ S-wave phase shifts according to:
\bea
\phi_0(s)&=&\theta(M-\sqrt{s}) \delta_0^0(s)+\theta( \sqrt{s}-M)(\pi-\delta_0^0(s))\nonumber\\
\phi_2(s)&=&\delta_0^2(s),
\eea
where $M$ is the mass of the $f_0$ resonance.  

The values used for the parameters entering the representations above are:
\bea
L_9^r+L_{10}^r&=&1.43\pm 0.27\times 10^{-3}\nonumber\\
s_i&=&2 (M_i^2-M_\pi^2)\nonumber\\
R_\omega&=& 1.35/GeV^2;~~~R_\rho=0.12/GeV^2
\eea

and the $\pi\pi$ phase shifts are well approximated up to $\sqrt{s}\sim 1.5 GeV$ by the parametrization:
\bea
\delta_0^I(s)&=& \arcsin\left(\frac{\Gamma_I}{2\sqrt{(\sqrt{s} - M_I)^2 +\frac{ \Gamma_I^2}{4}}}\right)+\sum_{n=0}^N a_n\;\, (\sqrt{s})^n
\eea
where we include one single resonance for each $I=0,2$.

For the available data we need only up to $N=3$ for $I=0$, with the result:
\bea
M_0=0.994 GeV;&&\Gamma_0=0.0624 GeV\nonumber\\
a_0=-1.439;&&a_1=6.461 /GeV;~~a_2=-5.529 /GeV^2;~~a_3=2.022 /GeV^3
\eea
For the case $I=2$ one finds that   the resonance term is not needed at all and a good fit is provided with $N=3$ with the result:
\beq
a_0=-0.878;&&a_1=-0.611 /GeV;~~a_2=-0.083 /GeV^2;~~a_3=0.115 /GeV^3
\eea


The $\gamma\gamma\to\pi_0\pi_0$ in the S-wave approximation
valid up to about $\sqrt{s}\sim 0.9 GeV$ is given by:
\bea
\sigma_{\gamma\gamma\to\pi^0\pi^0}(|\cos \theta|<Z)(s)&=&\frac{\pi \alpha_{EM}^2}{s^2}
\frac Z2  \sqrt{s (s -4 M_\pi^2)} \\
&\times&(\mid A(s) s-M_\pi^2  B(s)\mid^2\nonumber\\
&+&\frac {1}{s^2} \left(M_\pi^4- \frac{1}{16}(\frac{Z^2}{3} s(4 M_\pi^2-s)+4(s-2 M_\pi^2)^2)\right)\mid B(s)\mid ^2) \nonumber
\eea
 Fitting to the Crystal Ball data \cite{Marsiske:1990h} the parameters $c_0,\;d_0,\;c_2,\;d_2$ can be estimated,  giving in the corresponding units: 
\bea
c_0&=& -0.529 \nonumber\\
d_0&=&-2.033 \nonumber\\
c_2&=& 0.953\nonumber\\
d_2&=& -1.271.
\eea


 



\end{document}