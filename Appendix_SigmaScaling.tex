\section{Scale factors for Primakoff, nuclear coherent, and nuclear incoherent cross sections  \label{sec:SigmaScaling}}
Fig.~\ref{fig:leaddndt} shows $^{208}$Pb data from the PrimEx experiment \cite{Larin:2010kq}.   NPP will run on the same target and the same approximate incident beam energy, $\approx 6$ GeV, as PrimEx.   Using known analytical forms for the processes shown in the figure, known photo-nuclear cross sections, and estimates for nuclear attenuation from the PrimEx $^{208}$Pb analysis, we estimate numerical factors for scaling the Primakoff, nuclear coherent and nuclear incoherent total cross sections seen in PrimEx to the conditions for NPP. 
    

 \subsection{Scale factor for the Primakoff cross section}
The standard equation for Primakoff $\pi^0$ production is given by: 

$$ { d^2 \sigma_{PrimEx} \over d \Omega } = \Gamma_{\pi^0 \rightarrow \gamma \gamma} { 8 \alpha Z^2 \over M^3_{\pi} }
{ \beta^3 E_\gamma^4 \over Q^4 } F^2_{EM}(Q^2) \sin^2 \theta   $$ 
where $\Gamma_{\pi^0 \rightarrow \gamma \gamma} = 7.7$ eV is the $\pi^0$ radiative width.  The cross section for Primakoff $\pi \pi$ production with $P_{\gamma}=0$ is given by: 
$$ {d^2 \sigma_{NPP} \over d \Omega_{\pi \pi} dM_{\pi \pi} } = {2 \alpha Z^2 \over \pi^2} 
{E_\gamma^4 \beta^2 \over M_{\pi \pi} } {\sin^2 \theta \over Q^4 } F^2_{EM}(Q^2) 
 \sigma( \gamma \gamma \rightarrow \pi \pi) $$
The above equation can be reorganized so that it has a structure similar to the standard Primakoff equation: 
$$ {d^2 \sigma_{NPP} \over d \Omega_{\pi \pi} } \approx
\Big[ {1 \over 4 \pi^2 }  {M^2_{\pi \pi} \over \beta }  \sigma( \gamma \gamma \rightarrow \pi \pi) \Delta M_{\pi \pi} \Big]  { 8 \alpha Z^2 \over M^3_{\pi \pi} }
{ \beta^3 E_\gamma^4 \over Q^4 } F^2_{EM}(Q^2) \sin^2 \theta  $$
$$ {d^2 \sigma_{NPP} \over d \Omega_{\pi \pi} } \approx
\Gamma_{\pi^0 \pi^0 \rightarrow \gamma \gamma}  { 8 \alpha Z^2 \over M^3_{\pi \pi} }
{ \beta^3 E_\gamma^4 \over Q^4 } F^2_{EM}(Q^2) \sin^2 \theta  $$
where  $\Gamma_{\pi^0 \pi^0 \rightarrow \gamma \gamma}$ is the effective radiative width for $\pi^0 \pi^0 \rightarrow \gamma \gamma$, 
$$ \Gamma_{\pi^0 \pi^0 \rightarrow \gamma \gamma} =\Big[ {1 \over 4 \pi^2 }  {M^2_{\pi \pi} \over \beta }  \sigma( \gamma \gamma \rightarrow \pi \pi) \Delta M_{\pi \pi} \Big] $$
Taking $M_{\pi \pi} \approx 0.4$ GeV, $\Delta M_{\pi \pi}\approx 0.4$ GeV, and $\sigma ( \gamma \gamma \rightarrow \pi^0 \pi^0) \approx 10$ nb gives, 
$$\Gamma_{\pi^0 \pi^0 \rightarrow \gamma \gamma} \approx 42\ eV$$

The angular dependence of the Primakoff differential cross section for single and double-pion production is given by,
$$ {d\sigma \over d \Omega} \sim {\sin^2 \theta \over Q^4}|F(Q^2)|^2$$
It can be shown (see notes from R. Miskimen) that the peak of the Primakoff differential cross section is at the angle,
$$ \theta_{\rm max}={s \over 2 E^2_\gamma}$$
where s is the invariant mass squared of the $\pi$ or $\pi \pi$ system.   

The total cross section is given by the integral

$$ \sigma \sim \int^\Theta_0 {\sin^2 \theta \over Q^4}|F(Q^2)|^2 2 \pi \sin \theta d\theta$$
where $\Theta$ is the upper integration limit. Working in the small angle limit the total cross section is,
$$\sigma \sim 2\pi \int^\Theta_0 
{\theta^3 \over \left( {s^2 \over 4E^2_\gamma}+ E^2_\gamma \theta^2  \right)^2} 
\Big[ 1-{1 \over 6}<r^2>_{\rm charge}\left( {s^2 \over 4E^2_\gamma}+ E^2_\gamma \theta^2  \right) +... \Big]^2
d \theta$$
$$\sigma \sim 2\pi \int^\Theta_0 
{\theta^3 \over \left( {s^2 \over 4E^2_\gamma}+ E^2_\gamma \theta^2  \right)^2} 
\Big[ 1-{1 \over 3}<r^2>_{\rm charge}\left( {s^2 \over 4E^2_\gamma}+ E^2_\gamma \theta^2  \right) \Big]
d \theta$$
Setting $\Theta = C\theta_{\rm max}$ the integral can be evaluated giving, 
$$ \sigma \sim {\pi \over E^4_\gamma}
\left[ \Big( \ln(1+C^2)-{C^2 \over 1+C^2}  \Big)
- {s^2 \over 4 E^2_\gamma}  { <r^2_{\rm charge}> \over 3} 
\Big(C^2 - \ln(1+C^2) \Big) \right]$$

The integrand in the integral goes through zero at
$$\theta_{\rm min} \approx {1 \over E_\gamma}\sqrt{3 \over <r^2>_{\rm charge}}$$
provded that,
$$ {1 \over 3} <r^2>_{\rm charge}{s^2 \over 4 E^2_\gamma} << 1$$
which is valid for CPP and NPP. The angle $\theta_{\rm min}$ is approximately the first minimum of the nuclear charge form factor. 
Setting the upper limit of integration to  $\Theta = C \theta_{\rm max} = \theta_{\rm min}$ gives 
$$C= {2 E_\gamma  \over s}\sqrt{3 \over <r^2>_{\rm charge}}$$

$$ \sigma \sim {\pi \over E^4_\gamma}
\left[ \Big( \ln(1+C^2)-{C^2 \over 1+C^2}  \Big)
- 
\Big(1 - {1 \over C^2} \ln(1+C^2) \Big) \right]$$
The $1/E^4_\gamma$ factor cancels the $E^4_\gamma$ in the Primakoff equation, and therefore the energy dependence of the integrated cross section is relatively weak, given by the dependence of the above equation on  $E_\gamma \sim C$.  

For the $\pi \pi (\pi)$ final state C = 4.7 (40.8), and the ratio of integrals for double pion to single pion Primakoff production is approximately $I_{\pi\pi} / I_{\pi} \approx 0.24$
The final result for the ratio of Primakoff cross sections $\sigma_{\pi^0 \pi^0}$ to $ \sigma_{\pi^0}$ is given by, 
\begin{eqnarray}
\sigma_{\pi^0 \pi^0}   \Big/ \sigma_{ \pi^0}   
\approx 
\Big(\Gamma_{\pi^0 \pi^0 \rightarrow \gamma \gamma}
\Big/ \Gamma_{\gamma \gamma} \Big)
\times \Big( M_\pi \Big/ 
M_{\pi \pi}\Big)^3\times \Big( I_{\pi \pi} \Big/ I_\pi \Big)
\approx 0.050  \label{eq:sigpipi_over_sigpi}
\end{eqnarray}



 \subsection{Scale factor for the nuclear coherent cross section}

The nuclear coherent cross section for  $\pi^0$ photo-production is given by: 

$$ {d\sigma_{\gamma A \rightarrow A  \pi^0 } \over dt } \approx \eta A^2 { d\sigma_{\gamma N \rightarrow N \pi^0 } \over dt }\sin^2 \theta F^2(t) $$
where $\eta$ is the nuclear absorption factor for $\pi^0$ production, A is the atomic mass number, $d\sigma_{\gamma N \rightarrow N\pi^0 } / dt$ is the $\pi^0$ photo-production cross section on the nucleon, and $F(t)$ is the nuclear matter formfactor.  The nuclear coherent cross section for  $\pi^0 \pi^0$ photo-production has a similar form: 

$$ {d\sigma_{\gamma A \rightarrow A  \pi^0 \pi^0} \over dt } \approx  \eta^2 A^2 { d^2 \sigma_{\gamma N \rightarrow N \pi^0 \pi^0} \over dt dM_{\pi \pi}}\Delta M_{\pi \pi} \sin^2 \theta F^2(t) $$
where $d^2\sigma_{\gamma \rightarrow N\pi^0 \pi^0} / dt dM_{\pi \pi}$ is the $\pi^0 \pi^0$ photo-production cross section on the nucleon.
  In the near threshold region the dominant channel for $\pi^0 \pi^0$ is through f$_0(500)$ photo-production.    Cross sections for f$_0(500)$   have been measured in 
  $\gamma p \rightarrow \pi^+ \pi^-$ at 3.6-3.8 GeV \cite{Battaglieri:2009aa}.   The s-wave t and $M_{\pi^+ \pi^-}$ distributions are shown in Fig. \ref{f0_500}, the former at M$_{\pi \pi}=0.4$ GeV, and the latter at $t=0.5 GeV^2$.  Note that  
  $d\sigma^2 / dt dM_{\pi^+ \pi^-}$ is relatively flat versus M$_{\pi \pi}$ in the threshold region. The relevant cross section for this analysis is $d\sigma^2 / dt dM_{\pi^+ \pi^-}|_{ t = 0} \approx 1.0 \mu b/{\rm GeV}^3$  multiplied by an isospin factor of 1/2 to account for the f$_0(500)$ branching fraction to $\pi^0 \pi^0$, giving $d \sigma_{\gamma N \rightarrow N \pi^0 \pi^0} / dt dM_{\pi \pi} \approx 0.5 \mu b / {\rm GeV}^2 $. The cross section for  $\gamma p \rightarrow  p \pi^0$ at 6 GeV has been measured at SLAC \cite{Anderson:1971}, with cross sections shown in Fig. \ref{SLAC};   $d\sigma / dt|_{t=0} \approx 1.5 \mu b/{\rm GeV}^2$ .   Estimates from the PrimEX $^{208}$Pb analysis give $\eta \approx 0.55$.  
  
  Assuming an exponential behavior for the differential cross section on the nucleon, 
  $$ {d\sigma_{\gamma N \rightarrow N X} \over dt} = \sigma_0 e^{-kt}$$
  the coherent angular distribution is given by
  $$ { d\sigma_{\rm coherent} \over d\Omega} = \sigma_0 \sin^2 \theta |F(t)|^2 e^{-kt} {dt \over d\Omega}$$ 
  For CPP and NPP we have, 
  $$ {2k \over <r^2>_{\rm matter}}<<1 $$
  and 
  $${1 \over 6}<r^2>_{\rm matter} {s^2 \over 4E^2_\gamma}<<1$$
   Then to leading order in $\theta^2$ the peak coherent cross section is at angle 
$$ \theta_{\rm max}={1 \over E_\gamma}\sqrt{ 2 
 \over <r^2>_{\rm matter} } $$
 
 
  The total cross section is given by, 
  $$ \sigma_{\rm coherent} \sim \sigma_0 \int^{C\theta_{\rm max}}_0 \sin^2 \theta |F(t)|^2 e^{-kt} {dt \over d\Omega} 2\pi \sin\theta d\theta$$ 
  where C is a dimensionless scale parameter. Working in the small angle approximation
  and using the condition, 
 $$k\big( {s^2 \over 4 E^2_\gamma} \big) <<1$$
 we can reduce the total cross section  to this form, 
  $$ \sigma_{\rm coherent} \sim {8\sigma_0 \over <r^2_{\rm charge}>^2E^2_\gamma}
  \int^C_0 y^3 \Big[1- {1 \over 3} y^2 \Big]^2 dy  $$ 
  Since the integral depends only on C, the integral cancels out in taking the ratio
  $\sigma_{\gamma A \rightarrow A \pi^0 \pi^0}   \Big/  \sigma_{\gamma A \rightarrow A \pi^0} $. 
  
  Here we give our final result for the ratio of  $\pi^0 \pi^0$ to $\pi^0$  coherent cross sections:

$$  \sigma_{\gamma A \rightarrow A \pi^0 \pi^0}   \Big/  \sigma_{\gamma A \rightarrow A \pi^0}  \approx \eta 
 {d^2\sigma_{\gamma N \rightarrow N \pi^0 \pi^0} \over dt dM_{\pi \pi} } \Delta M_{\pi \pi} \Big/ { d\sigma_{\gamma N \rightarrow N \pi^0} \over dt} \approx 0.07 $$

\subsection{Scale factor for the nuclear incoherent cross section}

The incoherent cross section for  $\pi^0$ photo-production is given by: 

$$ {d\sigma_{\gamma A \rightarrow   \pi^0 } \over dt } \approx \eta A \Big(1 -G(t) \Big){ d\sigma_{\gamma N \rightarrow N \pi^0 } \over dt } $$
where G(t) is a Pauli suppression factor, with G(0)=1 and $G(t)\rightarrow 0$ for $t> k_F$, where $k_F$ is the nuclear Fermi momentum, $k_F\approx 260$ MeV/c. The incoherent cross section for  $\pi^0 \pi^0$ photo-production has a similar form: 
$$ {d\sigma_{\gamma A \rightarrow   \pi^0 \pi^0 } \over dt } \approx \eta^2 A \Big(1 -G(t) \Big){ d^2\sigma_{\gamma N \rightarrow N \pi^0 \pi^0 } \over dt dM_{\pi \pi}} \Delta M_{\pi \pi} $$
The photo-production cross sections on the nuclon are identical to those used in the estimation of the coherent cross section.   

The total cross section is given by, 
$$\sigma \sim \sigma_0 \int^{C E^2_\gamma \theta^2_{\rm max}}_{s^2/4E^2_\gamma} \Big( 1 - G(t) \Big) e^{-kt} dt  $$
where we integrate up to a multiple of $E_\gamma^2 \theta_{\rm max}^2$, where $\theta_{\rm max}$  is the angle of the peak coherent cross section, and $s^2/4E^2_\gamma$ is the minimum t in the reaction.  
Because $s^2/4E^2_\gamma << k^2_F$ and $1-G(s^2/4E^2_\gamma) \approx 0$, the lower limit on the integral can be replaced with zero, 
$$\sigma \sim \int^{CE^2_\gamma \theta^2_{\rm max}}_0 \Big( 1 - G(t) \Big) e^{-kt}  dt $$
For $\pi$ or $\pi \pi$ photo-production 
$$ C  E^2_\gamma \theta^2_{\rm max} k \approx C  { 2 \over <r^2>_{\rm matter} } k << 1 $$
provided that $C \le 5$. In this case we can neglect the exponential in the integral, 
$$\sigma \sim \int^{C E^2_\gamma \theta^2_{\rm max}}_0 \Big( 1 - G(t) \Big) dt $$
Since the integral depends only on the upper integration limit,   the integral  cancels out in taking the ratio $ \sigma_{\gamma A \rightarrow  \pi^0 \pi^0}  \Big/ \sigma_{\gamma A \rightarrow  \pi^0} $. 
Our final result for the ratio of  $\pi^0 \pi^0$ to $\pi^0$ incoherent cross sections is given by: 

$$  \sigma_{\gamma A \rightarrow  \pi^0 \pi^0}  \Big/ \sigma_{\gamma A \rightarrow  \pi^0}  \approx \eta 
 {d^2\sigma_{\gamma N \rightarrow N \pi^0 \pi^0} \over dt dM_{\pi \pi} }  \Delta M_{\pi \pi}\Big/ { d\sigma_{\gamma N \rightarrow N \pi^0} \over dt} \approx 0.07 $$

\subsection{Summary}
NPP will take data on the same target, $^{208}$Pb, and the same approximate incident beam energy, $\approx 6$ GeV, as PrimEx. Using known analytical expressions for the Primakoff, nuclear coherent, and nuclear incoherent cross sections, known photo-nuclear cross sections, and estimates for nuclear attenuation from the PrimEx $^{208}$Pb analysis, numerical factors are calculated for scaling the total cross sections seen in PrimEx to the 
conditions for NPP.  Assuming $M_{\pi^0 \pi^0} \approx 0.4$ GeV, and a width  $\Delta M_{\pi^0 \pi^0} \approx 0.4$ GeV, the scale factors for Primakoff, nuclear coherent, and nuclear incoherent total cross sections are approximately $\times 0.05$, $\times 0.07$, and $\times 0.07$, respectively. 

\begin{figure}
\centering
\includegraphics[width=6in]{figures/f0_500.png}
\caption{CLAS data for s-wave $\pi^+ \pi^-$ photoproduction on the proton at $3.2 < E_\gamma < 3.4$ GeV \cite{Battaglieri:2009aa}. }
\label{f0_500}
\end{figure}

\begin{figure}
\centering
\includegraphics[width=4in]{figures/SLAC_data.pdf}
\caption{SLAC data for $\pi^0$ photo-production on the proton \cite{Anderson:1971}.}
\label{SLAC}
\end{figure}
