\documentclass[letterpaper,12pt]{article}
\usepackage{amsmath}
\usepackage{graphicx}
\usepackage{epstopdf}
\usepackage{url}
\usepackage{multirow}
\usepackage{authblk}
\usepackage{color}  
\usepackage{enumitem}
\def\red{\color{red}}
\def\blue{\color{blue}}
\def\green{\color{green}}
\def\violet{\color{\frac{\lambda^a}{2}iolet}}
\def\orange{\color{orange}}

\usepackage[margin=0.65in]{geometry}

% Uncomment the following two lines to have line numbers printed in the document
\usepackage{lineno}
%\linenumbers

% Make author affiliations list small and italic
\renewcommand\Authfont{\footnotesize}
\renewcommand\Affilfont{\itshape\small}

\newcommand{\gx}{\textsc{GlueX}}


\date{August 22, 2019}

\title{\Large \textbf{Measuring the Neutral Pion Polarizability}\\
\large{***DRAFT*** Proposal Submitted to PAC ?? ***DRAFT***}
}
% {\small  (A proposal to the 40$^\mathrm{th}$ Jefferson Lab Program Advisory Committee)}}


% This author list modified for the CPP proposal

%\title{Authors With Multiple Affiliations}
%\affiliation{First Institution/Department, Affiliation, City, Country }{FIRSTAFF}
%\affiliation{Second Institution/Department, Affiliation, City, Country }{SECONDAFF}
%\author{A.B. Firstauthor}{FIRSTAFF}
%\author{C. Coauthor}{SECONDAFF}
%\author[2]{D.E. Othercoauthor}{FIRSTAFF}

% \author{E.~Chudakov\thanks{Hall D Leader}}
% \author[1]{M.M.~Ito\footnote{Spokesperson}}
\author[1]{M.M.~Ito}
\author[1,2]{J.~Goity}
\author[1]{D.~Lawrence}
% \author[1]{D.~Lawrence$&^\ast$}
% \author[1]{E.S.~Smith$^\ast$\footnote{Contact Person, email: elton@jlab.org}}
\author[1]{E.S.~Smith\footnote{Contact Person, email: elton@jlab.org}}
\author[1]{B.~Zihlmann}

\author[3]{R.~Miskimen}
%\author[3]{R.~Miskimen$^\ast$}
\author[3]{I.~Larin}

\author[4]{A.~Aleksejevs}
\author[4]{S.~Barkanova}

\author[5]{A.~Austregesilo}

\author[6]{D.~Hornidge}

\author[7]{P.~Martel}

\affil[1]{Thomas Jefferson National Accelerator Facility, Newport News, VA}
\affil[2]{Hampton University, Hampton, VA}
\affil[3]{University of Massachusetts, Amherst, MA}
\affil[4]{Memorial University of Newfoundland, Corner Brook, Canada}
\affil[5]{Carnegie Mellon University, Pittsburgh, PA}
\affil[6]{Mount Allison University, Sackville, New Brunswick, Canada}
\affil[7]{Institute for Nuclear Physics, Johannes Gutenberg University, Mainz, Germany}








%\collaboration{The \textsc{GlueX}~Collaboration}


\begin{document}
\setlength{\parindent}{0in}

% Set narrow margins to allow full collaboration list to appear on single page
%\newgeometry{top=0.in,bottom=0.5in,right=0.5in,left=0.5in}

% Main title
\maketitle

% This prevents the page number from being printed on the title page. We do this
% because the bottom margin is so small the page number would otherwise be
% too far down.
\thispagestyle{empty}

% Set back to nominal margins (wider ones were set just for collaboration list)
\newgeometry{top=1.5in,bottom=1.5in,right=1.0in,left=1.0in}

%---------------------------------------------------------------------------------------------------------------
%--------------------------------- Abstract ------------------------------------------------------------------
%---------------------------------------------------------------------------------------------------------------
%\section{Abstract}
\begin{abstract}
This proposal presents our plan to make a precision measurement of the cross section for $\gamma \gamma \rightarrow \pi^0 \pi^0$ via the Primakoff effect using the GlueX detector in Hall D. The aim is to significantly improve the data in the low $\pi^0\pi^0$ invariant mass domain, which is essential for understanding the low energy regime of Compton scattering on the $\pi^0$. In particular, the aim is to obtain a first accurate experimental determination of the neutral pion polarizability $\alpha_\pi - \beta_\pi$, which is one of the important predictions of chiral perturbation theory and a key test of chiral dynamics on the $\pi^0$.  Our goal is to measure $\sigma(\gamma\gamma\rightarrow\pi^0\pi^0)$ to a precision of $\sim$1\%, which would determine the combination of $\alpha_{\pi^0}-\beta_{\pi^0}$ to a precision of 10\%. We expect this experiment to run concurrently with the previously approved experiment to measure the charged pion polarizability (CPP) \cite{CPPexp} in Hall D.
\end{abstract}

\parindent 15pt
\parskip 8 pt

%---------------------------------------------------------------------------------------------------------------
%--------------------------------- Introduction ------------------------------------------------------------------
%---------------------------------------------------------------------------------------------------------------
\section{Introduction}

Electromagnetic polarizabilities are fundamental properties of
composite systems such as molecules, atoms, nuclei, and hadrons
\cite{Holstein:1990qy}. Whereas form factors provide information
about the ground state properties of a system, polarizabilities
provide information about the excited states of the system, and are
  therefore determined by the system's dynamics.
%For atomic systems polarizabilities are on the order of the atomic
%volume.  For hadrons the polarizabilities are much smaller than the
%volume, typically of the order of $10^{-4}$ fm$^3$, because of the
%greater stiffness of the QCD force as compared to the electromagnetic
%force.
Measurements of hadron polarizabilities provide an important test
point for Chiral Perturbation Theory, dispersion relation approaches,
and lattice calculations. Among the hadron polarizabilities, the
neutral pion polarizability is important because it
tests fundamental symmetries, in particular chiral symmetry and its
realization in QCD.  Indeed, the non-trivial (non-perturbative) vacuum
properties of QCD result in the phenomenon of spontaneous chiral
symmetry breaking, giving rise to the Goldstone boson nature of the
pions.  In particular, the Goldstone boson nature of the $\pi^0$
manifests itself most notably in its decay into $\gamma\gamma$ and
also in its electromagnetic polarizability, which according to ChPT
can be predicted to leading order in the expansion in quark
masses.


Hadron polarizabilities are best measured in Compton scattering
experiments where, in the case of nucleon polarizabilities, one looks
for a deviation of the cross section from the prediction of Compton
scattering from a structureless Dirac particle.
In the case of pions, the long lifetime of the charged pion permits
experiments of low energy Compton scattering using a beam of high
energy pions scattering on atomic electrons. On the other hand, the
short lifetime of the neutral pion requires an indirect study of low
energy Compton scattering via measurements of the process $\gamma
\gamma \rightarrow \pi^0 \pi^0$, a method that can also be applied to
the charged pion (CPP) and for which a proposal in Hall D is already
approved \cite{CPPexp}.

Measurements of hadron polarizabilities are among the most difficult
experiments performed in photo-nuclear physics. For charged hadrons,
because of the Born term, the polarizability effect in the cross
section can range from 10 to 20\% depending on the kinematics.  For
neutral hadrons, where the Born term is absent, the polarizability
effect will be much less than this.  To set reasonable expectations
for what can be accomplished in a measurement of this type, it is
important to recognize that after 30 years of dedicated experiments
using tagged photons at facilities across North America and Europe,
the error on the proton electric polarizability is 4\%, without doubt
the paramount experimental achievement in this field. However, the
error on the proton magnetic polarizability is 16\%
\cite{PDGTanabashi:2018oca}.  Absolute uncertainties provide a better
gauge of a measurements sensitivity; for proton electric and magnetic
polarizabilities the uncertainty in both is $\pm 0.4 \times
10^{-4}~\mathrm{fm}^3$.  Another level of precision to consider for
setting expectations is the result COMPASS obtained for charged pion
$\alpha - \beta$. COMPASS provides also a Primakoff measurement. COMPASS
achieved a relative error of 46\% in $\alpha - \beta$ and an absolute
error of $\pm 0.9 \times 10^{-4}~\mathrm{fm}^3$.  COMPASS cannot
measure the neutral pion polarizability.



This proposal presents a plan to make a precision measurement of the
$\gamma \gamma^* \rightarrow \pi^0 \pi^0$ cross section using the
GlueX detector in Hall D.  The measurement is based on the Primakoff
effect which allows one to access the low $W_{\pi^0\pi^0}$ invariant
mass regime with the virtual photon $\gamma^*$ provided by the
Coulomb field of the target. The central aim of the measurement is to
drastically improve the determination of the cross section in this
domain, which is key for constraining the low energy Compton amplitude
of the $\pi^0$ and thus for extracting its polarizability.  At
present, the only accurate measurements exist for invariant masses of
the two $\pi^0$s above 0.7 GeV, far above the threshold 0.27 GeV. The
existing data at low energy were obtained in $e^+ e^- \to \pi^0\pi^0 $
scattering in the early 1990's with the Crystal Ball detector at the
DORIS-II storage ring at DESY \cite{Marsiske:1990hx}.

Meanwhile, theory has made significant progress over time, with
studies of higher chiral corrections 
\cite{Bellucci:1994eb,Gasser:2005ud,Aleksejevs:2014eea} and with the
implementation of dispersion theory analyses which serve to make use
of the higher energy data
\cite{Oller:2008kf,Dai:2014zta,Dai:2014lza,Moussallam:2013una}. It is
expected that the experimental data from this proposal, together with
these theoretical frameworks, will allow for the most accurate
extraction of the $\pi^0$ polarizabilities to date.


%---------------------------------------------------------------------------------------------------------------
%---------Theoretical predictions for the charged pion polarizability --------------------------
%---------------------------------------------------------------------------------------------------------------
\section{Theoretical predictions for the neutral pion polarizability}
The low energy properties of pions are largely determined by  their nature as
the Goldstone Bosons of spontaneously broken chiral symmetry in QCD,
and are described in a model independent way by the framework of
Chiral Perturbation Theory (ChPT) (Gasser and Leutwyler
\cite{Gasser:1983yg}), which implements a systematic expansion in low
energy/momentum and in quark masses.  In particular the pions' low
energy electromagnetic properties can serve as tests of their
Goldstone Boson (GB) nature. One such a case is the $\pi^0\to
\gamma\gamma$ decay, which at the same time tests its GB nature and
the chiral anomaly. Another case is low energy Compton scattering on
pions: at low energy the Compton differential cross section can be
expanded in powers of the photon energy and expressed in terms of the
corresponding pion charge form factor (for charged pion) and the electric
and magnetic
polarizabilities, where the latter give the order $\omega^2$ terms in
the Compton cross section, being $\omega$ the photon energy. The
polarizabilities appear as deviations
of the pions from point like particles, and thus result from carrying
out the chiral expansion to the next-to-leading order. For both
charged an neutral pions the polarizabilities are fully predicted at
leading order in quark masses, and thus represent a sensitive test of
chiral dynamics. For the charged pions, at $O(p^4)$ ChPT predicts that
the electric and magnetic polarizabilities ($\alpha_{\pi^+}$ and
$\beta_{\pi^+}$) are related to the charged pion weak form factors
$F_V$ and $F_A$ in the decay $\pi^+ \rightarrow e^+ \nu \gamma$
\begin{equation}\label{alpha_and_beta}
\alpha_{\pi^+} = -\beta_{\pi^+} \propto \frac{F_A}{F_V} = \frac{1}{6} ( l_6 - l_5 ),
\end{equation}

\noindent where $l_5$ and $l_6$ are low energy constants in the Gasser
and Leutwyler effective Lagrangian \cite{Gasser:1983yg}.  Using recent
results from the PIBETA collaboration for $F_A$ and $F_V$
\cite{Bychkov:2008ws}, the $O(p^4)$ ChPT prediction for the charged
pion electric and magnetic polarizabilities is given by:
\begin{equation}\label{one_loop}
\alpha_{\pi^+} = -\beta_{\pi^+} = (2.78 \pm 0.1) \times 10^{-4}~{\rm fm}^3.
\end{equation}
\begin{figure}[h]
\begin{center}
\includegraphics[height=2cm,angle=0]{figures/Fig-TP-1.pdf}
%\includegraphics[height=6cm,width=8cm,angle=0]{gA-Mpi-LHPC-band.pdf}
\end{center}
\caption{
\label{fig:digrams}}
\end{figure}


In the case of the neutral pion, the polarizabilities are determined
by the one loop chiral contributions (see Fig.\,\ref{fig:digrams})
which are calculable, free of unknown parameters, and given only in
terms of the fine structure constant, the pion mass and the pion decay
constant:
\begin{eqnarray}
\alpha_{\pi^0}+\beta_{\pi^0}&=&0\nonumber\\
\alpha_{\pi^0}-\beta_{\pi^0}&=&-\frac{\alpha}{48 \pi^2 M_\pi F_\pi^2}\simeq -1.1\times 10^{-4}~{\rm fm}^3
\end{eqnarray}
However, there is a range of predictions beyond
NLO and the experimental test of these important predictions is very
challenging. In the first place, the polarizabilities drive the very
low energy regime of Compton scattering on the $\pi^0$ as there is no
Thomson term, so one would expect that it would be easier to determine
them than in the charged pion case.  However, in the first place
Compton scattering on the $\pi^0$ is experimentally inaccessible due
to its short lifetime, and therefore it is necessary to resort to the
process $\gamma\gamma\to \pi^0\pi^0$ of this proposal. In addition, ChPT
indicates that the
polarizabilities are smaller in the case of the neutral pion, about a
third of their value for the charged pion, i.e. somewhere between
$-1.7\times 10^{-4}~{\rm fm}^3$ and $-1.9\times 10^{-4}~{\rm fm}^3$,
depending on the model used to estimate higher order effects in the chiral
expansion. The challenge is therefore to measure the
cross section for $\gamma\gamma \to \pi^0\pi^0$ with sufficient
accuracy at low invariant mass $W_{\pi\pi}$ so that one can infer the
low-energy Compton amplitude and extract the polarizabilities. The demand for
accuracy is required in order to allow for the extrapolation of the Compton
amplitude from the kinematics of $\gamma\gamma \to \pi^0\pi^0$ to   low energy
Compton scattering, something that is at present impossible with the poor
accuracy of the only available data from the Crystal Ball experiment \cite{}.
 


For
this purpose, the theoretical foundations have been laid in works
studying $\gamma\gamma\to \pi^0\pi^0$ both using ChPT (Bellucci et al
\cite{Bellucci:1994hx,Bellucci:1994eb}, Gasser et al
\cite{Gasser:2005ud}, Aleksejevs and Barkanova
\cite{Aleksejevs:2014eea}) and dispersion theory (Oller and Roca
\cite{Oller:2008kf}, Dai and Pennington
\cite{Dai:2014zta,Dai:2014lza}, Moussallam
\cite{Moussallam:2013una}). In particular, in ChPT at the next-to-next
to leading order, which provides the higher order quark mass
corrections to the polarizabilities, some of the low energy constants
need to be fixed and for that a significantly more accurate
measurement of the $\gamma\gamma\to \pi^0\pi^0$ cross section is
needed than available presently.

Accurate measurements of the cross section near threshold combined
with data for $W_{\pi\pi}>0.6$ GeV will provide the necessary input
for performing a full theoretical analysis, combining dispersion
theory with and without inputs from ChPT at low energy. This is a well
established method which has been used to analyze $\pi\pi$
scattering and also to the very problem of the $\gamma\gamma \to \pi^0\pi^0$
process, where numerous works have been steadily improved the theoretical
dispersive analysis, to mention a
few \cite{Donoghue:1993kw,Oller:2007sh,Oller:2008k,Moussallam:2013una,Dai:2016ytz}.
Through such an analysis it will be possible to determine,
via combination with ChPT, the low energy Compton amplitude and
extract the   combination $\alpha_\pi-\beta_\pi$. The
latter extraction represents a challenge as shown in
Fig.\,\ref{fig:previousdata}, where the polarizabilities have a small
direct effect on $\gamma\gamma\to \pi\pi$.  Calculations by Dai and
Pennington (Table II) \cite{Dai:2016ytz} indicate that a 1.3\%
determination of $\sigma(\gamma\gamma\rightarrow\pi^0\pi^0)$ will
determine the combination of $\alpha_{\pi^0}-\beta_{\pi^0}$ to a
precision of 10\%. 
%The preliminary study done by Aleksejevs and Barkanova for this proposal in relativistic ChPT with SU(3) input indicate that sensitivity could be even better depending on specific kinematics; the full evaluation is in progress. 
In general, the
determination of the accuracy one can get for $\alpha_\pi-\beta_\pi$
based on a more accurate measurement as the one proposed here is still
an issue being currently studied theoretically, with J. L.  Goity and
A. Aleksejevs and S. Barkanova forming a group to take a lead on the project. 
At present a theoretical study based on the S-wave dominance below
$W_{\pi\pi}\sim 0.8 GeV$ and dispersion theory has allowed to represent the two
Compton amplitudes $A$ and $B$ in the physical domain of the experiment. The
study of the extrapolation to low energy Compton kinematics is under study, in
particular the issues related to the stability of the dispersive analysis. This
study is expected to provide a more accurate estimate on the sensitivity with
which the experiment will allow for the determination of the polarizability
$\alpha-\beta$.


%However, even 10\% precision would still be sufficient to difference between various models and facilitate extraction of low-energy constants, and thus a highly desirable result for theory.

\begin{figure}[ht]
\begin{center}
\includegraphics[height=5cm,angle=0]{figures/Fig-TP-2pr.pdf}\includegraphics[height=5cm,angle=0]{figures/XBall.pdf}\\
\includegraphics[height=6cm,angle=0]{figures/Fig-TP-5.pdf}\i
%\includegraphics[height=6cm,width=8cm,angle=0]{gA-Mpi-LHPC-band.pdf}
\end{center}
\caption{Left panel: experimental status; right panel: results from the 1990
XBall experiment. The lower panel shows the effect of $\pi^0$ polarizabilities
on the cross section ($\sqrt{s}=W_{\pi\pi}$) \cite{Moussallam:2013una} .
\label{fig:previousdata}}
\end{figure}

\section{Past Measurements}
Past measurements of the $\gamma\gamma\to \pi^0\pi^0$ cross section can be
summarized as follows:
\begin{enumerate}
    \item 
In the early 1990's measurements were made in $e^+e^-$ collisions at DESY
with the XBall detector at the DORIS-II storage ring, which are the only
available data for $W_{\pi\pi}<0.6{\rm~ GeV}$ \cite{Marsiske:1990hx}. 
\item In 2008-9, measurements were carried out by BELLE
for $0.6{\rm~ GeV}<W_{\pi\pi}<4.0 {\rm~ GeV}$ \cite{Mori:2007bu,Uehara:2008ep,Uehara:2009cka}. 
\end{enumerate}
As mentioned above, several works have made use of dispersion theory methods
(Oller and
Roca \cite{Oller:2008kf}, Dai and Pennington \cite{Dai:2016ytz}, and
in particular Moussallam \cite{Moussallam:2013una} who performed the
dispersive analysis where one of the photons has non vanishing
virtuality, which is particularly important for our case.) with those
available data. In particular these methods give results for the cross
section at small $W_{\pi\pi}$, but the poor accuracy of the data in
that region does not serve as a useful constraint that could improve
those analyses. On the other hand, the ChPT calculations carried out
at NNLO (Bellucci et al \cite{Bellucci:1994hx,Bellucci:1994eb} ) can
only be fitted to the low $W_{\pi\pi}$ data, and thus the uncertainty
in the determination of low energy constants is rather large. It is therefore
expected that accurate data at low $W_{\pi\pi}<0.6{\rm~ GeV}$ will
have a very big impact on both theoretical approaches, which together
may allow for an accurate description of the low energy Compton
amplitude, and for a first time experimental determination of the
polarizability.


%---------------------------------------------------------------------------------------------------------------
%---------Measurements of the charged  pion polarizability-------------------------------------
%---------------------------------------------------------------------------------------------------------------
%\input{PreviousMeasurements.tex}

%---------------------------------------------------------------------------------------------------------------
%---------Measurements of the charged  pion polarizability at Jefferson Lab Hall D-----
%---------------------------------------------------------------------------------------------------------------


 %<><><><><><><><><><><><><><><><><><><><><><><><><><><><><><><><><><><><><><><><><><><><><>
 % Experimental conditions
 %<><><><><><><><><><><><><><><><><><><><><><><><><><><><><><><><><><><><><><><><><><><><><>
\section{Experimental conditions}
The measurement of the neutral pion polarizability is expected to run
concurrently with the experiment to measure the charged pion
polarizability (CPP) \cite{CPPexp} in Hall D. Essentially all the
optimizations for that experiment are expected to improve the
sensitivity of this experiment also. We briefly summarize the
configuration for CPP, which is compared in
Table\,\ref{tab:ccp_config} to nominal GlueX running.  
 
The diamond radiator will be adjusted to set the coherent peak of the
photon beam between 5.5 and 6 GeV. This enhances the polarization
significantly and also the tagging ratio.  The experimental target
will be placed upstream of the nominal GlueX target by 64 cm (z=1\,cm
in the Hall D coordinate system). These changes benefit the present
experiment. In addition, the CPP experiment will add multi-wire
proportional chambers downstream for muon identification, but these do
not impact this measurement one way or another.
 
% \begin{landscape}
\begin{table}[t]
\caption{Configuration of the CPP experiment compared to nominal GlueX. We propose that this experiment run concurrently with CPP. Detectors not identified
in the table are assumed to be operated under the same conditions as in the nominal configuration.
\label{tab:ccp_config}
}
\begin{center}
\begin{tabular}{|l|c|c|c|c|c|c|c|c|}
\hline
\hline 
  Configuration  & GlueX I  & CPP/NPP   \\  \hline \hline
  Electron  beam energy  &   11.6 GeV   &  11.6 GeV   \\ \hline 
  Emittance   &   10$^{-8}$m rad   &  10$^{-8}$m rad   \\ \hline 
  Electron  current  &   150 nA   &  20 nA  \\ \hline
  Radiator thickness  &   50$\mu$m  &  50 $\mu$m diamond   \\ \hline 
  Coherent peak  &   8.4 -- 9.0 GeV   &  5.5 -- 6.0 GeV     \\ \hline 
  Collimator aperture  &  5 mm & 5 mm   \\ \hline  
  Peak polarization  &   35\%   &  72\%     \\ \hline 
  % Coherent/Incoherent ratio  &  0.068   &   0.32?   \\ \hline  
  Tagging ratio  &   0.6   &   0.72 \\ \hline  
  Flux 5.5-6.0 GeV  &      &   11 MHz \\ \hline   
  Flux 8.4-9.0 GeV  &  20 MHz    &    \\ \hline  
  Flux 0.3-11.3 GeV  &  367 MHz    &   74 MHz \\ \hline 
  Target position  &   65 cm   &  1 cm    \\ \hline 
  Target, length   &  H, 30 cm   &  $^{208}$Pb, 0.028 cm   \\
%                            &   2.1 g/cm$^2$, 0.033 X$_0$ & 0.69 g/cm$^2$, 0.05 X$_0$ \\ 
 \hline  
  Start counter & nominal & removed \\ \hline
  Muon identification  &  None   &   Behind FCAL    \\ \hline   
  \hline
\end{tabular}
\end{center}
\end{table}
%\end{

\subsection{Expected signal}
In order to estimate rates, resolution and acceptance due to the
Primakoff reaction on lead, $\gamma ^{208}\rm{Pb}\rightarrow \pi^0
\pi^0\, \rm{Pb}$, we take the reaction process to be the same as for
charged pion production and given in Eq.~8 of the Proposal for the
Charged Pion Polarizability experiment \cite{CPPexp}, which is
reproduced here for convenience:
\begin{eqnarray}
\frac{d^2\sigma}{d\Omega_{\pi\pi}dW_{\pi\pi}} = \frac{2\alpha Z^2}{\pi^2} \frac{E^4_\gamma \beta^2}{W_{\pi\pi}} \frac{\sin^2\theta_{\pi\pi}}{Q^4} |F(Q^2)|^2 \sigma(\gamma\gamma\rightarrow\pi^0\pi^0) (1+P_\gamma \cos{2\phi_{\pi\pi}}).   \label{eq:PrimakoffSignal}
\end{eqnarray}
The $\gamma\gamma$ cross section for charged pions has been
substituted with the cross section for neutral pions, namely
$\sigma(\gamma\gamma\rightarrow\pi^0\pi^0)$. In this expression,
$\Omega_{\pi\pi}$ is the solid angle in the laboratory frame for the
emission of the $\pi\pi$ system, $W_{\pi\pi}$ is the $\pi\pi$
invariant mass, Z is the atomic number of the target, $\beta$ is the
velocity of the $\pi\pi$ system, $E_\gamma$ is the energy of the
incident photon, $F(Q^2)$ is the electromagnetic form factor for the
target with final-state-interaction (FSI) corrections applied,
$\theta_{\pi\pi}$ is the lab angle for the $\pi\pi$ system,
$\phi_{\pi\pi}$ is the azimuthal angle of the $\pi\pi$ system relative
to the incident photon polarization, and $P_\gamma$ is the incident
photon polarization.\footnote{The expression for the cross section in
  terms of invariant quantities can be found in
  Ref.\,\cite{hdnote3186}.}
\begin{figure}[tph]
\centering
\includegraphics[page=1,width=4.75in]{figures/sigma_2pi0_figs.pdf}
\caption{Parameterization of the $\sigma(\gamma\gamma\rightarrow \pi^0\pi^0)$ cross section as a function
of the 2$\pi$ invariant mass compared to the data from Crystal Ball \cite{Marsiske:1990hx}.
\label{fig:sigma_2pi0_figs_1}}
\end{figure}

The cross section for $\sigma(\gamma\gamma\rightarrow\pi^0\pi^0)$ has
been measured by the Crystal Ball Collaboration
\cite{Marsiske:1990hx}, albeit with limited statistical precision. We
have parameterized the cross section for $W_{\pi\pi}<0.8$ GeV, which
is of specific interest to this program as shown in
Fig.~\ref{fig:sigma_2pi0_figs_1}. Using this parameterization and
Eq.~\ref{eq:PrimakoffSignal}, we can calculate the photoproduction
cross section on lead, which is shown in
Fig.~\ref{fig:sigma_2pi0_figs_2}.
\begin{figure}[tph]
\centering
\includegraphics[page=2,width=4.75in]{figures/sigma_2pi0_figs.pdf}
\caption{Primakoff cross section for $\gamma Pb \rightarrow Pb\, \pi^0 \pi^0$ using the parameterization of  $\sigma(\gamma\gamma\rightarrow \pi^0\pi^0)$ in the previous figure. The integrated cross section is 0.3\,$\mu$b/nucleus.
\label{fig:sigma_2pi0_figs_2}}
\end{figure}
The integrated cross section is $0.30 \pm 0.05~\mu$b/nucleus. The
uncertainty comes from the model dependence and was obtained by
comparing two different calculations using completely different
parameterizations for the nuclear form factor on lead, $F(Q^2)$. For
reference,
we note that the cross section for charged pions $(\pi^+\pi^-$)
production is 10.9\,$\mu$b, a factor of 30 larger.

The number of neutral-pion-Primakoff-signal events produced during 20
PAC days is shown in Fig.~\ref{fig:sigma_2pi0_figs_3}. The impacts of
detector trigger, acceptance and resolution are discussed in the next
section.
\begin{figure}[tph]
\centering
\includegraphics[page=3,width=4.75in]{figures/sigma_2pi0_figs.pdf}
\caption{Estimated production rate  for $\gamma \rm{Pb}\rightarrow \pi^0\pi^0\,\rm{Pb}$ as a function 2$\pi$ mass. For this calculation, it is assumed the detector has perfect resolution and has a linearly increasing efficiency from zero at threshold up to 0.4 at 0.34 GeV (see top right of Fig.\,\ref{fig:twopi_primakoff_DSelect_p1_W_100000_sum} ).
\label{fig:sigma_2pi0_figs_3}}
\end{figure}

\subsection{Detector resolution}
\label{sec:acceptance}
The response of the GlueX detector to neutral pion Primakoff events
was simulated using the standard GlueX generation and reconstruction
tools, but with the specific geometry for the CPP experiment. The schematic of the detector configuration is shown in
Fig.~\ref{fig:GlueX_cpp}. The primary differences between the
standard GlueX geometry and CPP are summarized in
Table\,\ref{tab:ccp_config}. For this experiment, the main differences
include a) coherent peak position at 5.5-6 GeV and re-positioning of
the microscope to cover the coherent peak, b) solid $^{208}$Pb at
z=1cm, and c) Start counter removed. For the CPP experiment, the
addition of muon identification chambers behind the FCAL is
critical. However, for neutral pions this addition plays no role
because the photons are detected in the FCAL. The GEANT4 simulation,\footnote{The initial simulations used GEANT3
and show similar results.}
which is used for these studies, includes most changes except for the
addition of the muon chambers, which are not needed. In addition, the
microscope geometry has not been modified and we use the tagger
hodoscope for the region of interest in the simulation. The slightly reduced
energy of the hodoscope relative to the microscope has little impact
and the gaps between counters are ignored when simulating the tagged
flux.
\begin{figure}[h!]
\centering
\includegraphics[width=5.25in]{figures/GlueX_cpp.pdf}
\caption{Diagram of the GlueX detector including the additional muon chambers for the CPP experiment.}
\label{fig:GlueX_cpp}
\end{figure}

The Primakoff signal was generated according to the cross section
described in the previous chapter, using the
\emph{gen\_2pi0\_primakoff} program, which is a modified version of the CPP event generator. By default, the production amplitudes are
symmetrized between the two identical $\pi^0$'s by AmpTools. One
hundred thousand events of Primakoff and nuclear coherent interactions 
(see Section\,\ref{sec:NCback}) were generated with $M_{\pi\pi}<$0.9 GeV.
We used random triggers from run 30401\footnote{Run 30401 is a low-intensity run for GlueX,
but represents considerably higher background than expected for this experiment.} to add tagger accidentals 
and random hits in the drift chambers. These events were fed to GEANT4 to track particles, and subsequently
processed using \emph{mcsmear} to simulate the detector response. The
simulated events were then analyzed using the GlueX event filter to
analyze the reaction $\gamma Pb \rightarrow \pi^0 \pi^0$ with a
missing Pb nucleus and to constrain the detected photon pairs to the
$\pi^0$ mass. Energy and momentum conservation is imposed on the reaction as well as the requirement that all photons originate from a common vertex. The output of the reconstruction, both kinematically
fit and ``measured" quantities, were available for inspection.

In the following we show various reconstructed quantities as well as
estimated resolutions. The distribution of generated photon energy and
the unconstrained reconstructed momenta of the two pions are shown in
Fig.~\ref{fig:EgP1P2_signal_DSelector}.
\begin{figure}[tph]
\centering
\includegraphics[width=6in]{figures/EgP1P2_signal_DSelector.pdf}
\caption{Left: Generated photon energies. The increased rate near 5.5 GeV is due to tagger accidentals. Center: Reconstructed momentum distribution of one $\pi^0$. Right: Reconstructed momentum distribution of the second $\pi^0$.
\label{fig:EgP1P2_signal_DSelector}}
\end{figure}
The missing mass, 2$\pi$ mass and $-t$ distributions are shown in Fig.~\ref{fig:MMMpipit_signal_DSelector}.
\begin{figure}[tph]
\centering
\includegraphics[width=6in]{figures/MMMpipit_signal_DSelector.pdf}
\caption{Left: Missing mass distribution minus the mass of the recoil nucleus. Center: Kinematically fit $2\pi$ mass distribution. Right: Kinematically fit $-t$ distribution.
\label{fig:MMMpipit_signal_DSelector}}
\end{figure}
The reconstructed momentum relative to its generated value is shown in
Fig.~\ref{fig:DeltapDeltaPhi_signal_DSelector}. The central peak of the
kinematically fit momentum is about 2\%, similar to that for charged
pions. However, there are long tails that will effect the
final reconstruction. The resolution of the azimuthal angle,
$\phi_{\pi\pi}$, between the production and the photon polarization
planes is quite poor owing to the fact that the pion pairs are
produced at very shallow angles. Nevertheless it is sufficient to
measure the asymmetry due to the photon beam polarization.
\begin{figure}[tph]
\centering
\includegraphics[width=6in]{figures/DeltapDeltaPhi_signal_DSelector.pdf}
\caption{Left: Difference between measured and generated momentum. Center: Difference between kinematically fit and generated momentum. The central peak has a width of about 2\%. Right: Difference between the kinematically fit azimuthal angle $\phi_{\pi\pi}$ and its generated value.
\label{fig:DeltapDeltaPhi_signal_DSelector}}
\end{figure}
The resolution of the 2$\pi$ invariant mass is shown in Fig.~\ref{fig:Resolution_Mpipittag_signal_DSelector}, along with the resolution of Mandelstam $-t$, and the reconstructed time resolution. The mass resolution is about 12~MeV.
\begin{figure}[tph]
\centering
\includegraphics[width=6in]{figures/Resolution_Mpipittag_signal_DSelector.pdf}
\caption{Left: Difference between kinematically fit and generated 2$\pi$ mass. The central 2$\pi$-mass $\sigma$ is about 12 MeV. Center: Difference between kinematically fit and generated -t. Right: Difference between kinematically fit and generated 2$\pi$ polar angle. The resolution $\sigma$ of the reconstructed angle is 0.1 degrees.
%Right: Timing resolution relative to the accelerator RF signal is about 0.35 ns.
\label{fig:Resolution_Mpipittag_signal_DSelector}}
\end{figure}

\subsection{Trigger and acceptance}
The Primakoff reaction will transfer all the energy of the beam into
four photons, which are going forward. All this energy will be
deposited in the FCAL, except for leakage down the beampipe. We expect
a simple trigger with an energy threshold in the FCAL should have very
high efficiency for any events that can be reconstructed:
the FCAL trigger with the total energy threshold around 1$\,$GeV can be used. To estimate trigger rate we used the same method as for TOF trigger rate \cite{TOFrate} extracted from the dedicated runs with high random trigger frequency. We used FCAL total energy greater than 1GeV deposited within 40$\,$ns excluding the most inner FCAL layer as a trigger condition. Fig.~\ref{fig:fcalrate} shows the values for LH2 and ``empty'' target configurations.
Since the proposed lead target is 1.7 times thicker than LH2 target (in rad. lengths), we used ``empty'' target rate plus the difference between LH2 and ``empty'' target rates scaled with the factor of 1.7. That gives the value $\sim$9$\,$kHz for 20$\,$nA beam current.

\begin{figure}[tph]
\centering
\includegraphics[width=3in]{figures/fcalrate_vs_beam.pdf}
\caption{Estimated FCAL trigger rate for 1$\,$GeV
total energy threshold for LH2 (solid squares) and ``empty'' (empty squares) targets.
\label{fig:fcalrate}}
\end{figure}

The acceptance of the signal events can be determined by comparing the
kinematically fit to the generated distributions. The generated and
kinematically fit 2$\pi$ mass, $\phi_{\pi\pi}$ and $-t$ distributions
are shown in
Fig.\,\ref{fig:twopi_primakoff_DSelect_p1_W_100000_sum}. The
reconstruction was described in the previous section.
\begin{figure}[tph]
\centering
\includegraphics[width=6in]{figures/twopi_primakoff_DSelect_test_File_100000_sum_PrimNC.pdf}
\caption{Top left: Generated and kinematically fit (accepted) 2$\pi$-mass distribution. Top right: Acceptance as a function of 2$\pi$ mass. The acceptance is about 40\% at threshold. Bottom left: Generated and kinematically fit azimuthal angle $\phi_{\pi\pi}$. Bottom right: Generated and kinematically fit $-t$ distribution. }
\label{fig:twopi_primakoff_DSelect_p1_W_100000_sum}
\end{figure}
The acceptance is quite high at about 40\%. However, there is also
significant slewing due to resolution in most variables of
interest.  The
relatively poor resolution in $\phi_{\pi\pi}$ results in dilution of
the measured azimuthal dependence, which will need to be adjusted
based on simulation. Finally the measured $-t$ resolution roughly reproduces the generated slope despite the smearing of high rate regions down to low rate
regions.


%<><><><><><><><><><><><><><><><><><><><><><><><><><><><><><><><><><><><><><><><><><><><><>
 % Backgrounds
 %<><><><><><><><><><><><><><><><><><><><><><><><><><><><><><><><><><><><><><><><><><><><><>
\section{Backgrounds}

% \begin{landscape}
\begin{table}[t]
\caption{Comparison of backgrounds for the single $\pi^0$ channel and the present study for 
determination of the signal in the $\pi^0\pi^0$ channel. The relative backgrounds for this experiment
are expected to be smaller than those for the single $\pi^0$ channel.
\label{tab:Primex_sigmas}
}
\begin{center}
\begin{tabular}{|l|c|c|}
\hline
\hline 
 Integrated Fraction & $\gamma\,Pb\to \pi^0\, Pb$  & $\gamma\,Pb\to \pi^0\pi^0\, Pb$ \\  
  ($\theta <$ 1.5 degrees)                    &       & (This study) \\  \hline
  Primakoff signal  &   1.0   & 1.0   \\ \hline 
  Nuclear Coherent (NC)  & 0.39  &   0.38   \\ \hline 
  Interference  & 0.12  &  0.17   \\ \hline 
  $\gamma p \rightarrow \eta p$, BR($\eta \rightarrow 3\pi^0$)  &   -- & 0.16   \\ \hline 
  Incoherent (IC)  &   0.02  & 0.06  \\
  \hline   
  \hline
\end{tabular}
\end{center}
\end{table}



 \begin{figure}[tbh]
\begin{center}
%\includegraphics[height=5cm,viewport=250 180 0 360,clip=true]{CPP_production3}
\includegraphics[height=3cm,clip=true]{figures/Diagram_Primakoff.png} \hspace{1cm}
\includegraphics[height=3cm,clip=true]{figures/Diagram_hadronic.png}
\caption{Sketch of coherent two-pion production. Left) Signal: Primakoff mechanism, Right) Backgrounds: Other production mechanisms.
\label{fig:Diagram}}
\end{center}
\end{figure}

 \begin{figure}[tbh]
\begin{center}
\includegraphics[height=5cm,clip=true]{figures/Primex_sigmas_c1.png} \hspace{1cm}
\caption{Integrated angular distribution for the $\pi^0$ in $\gamma\,Pb\rightarrow \pi^0\,Pb$ as a function of the upper limit of integration.
\label{fig:Primex_sigmas_c1}}
\end{center}
\end{figure}


We first classify the various backgrounds and then describe them in more detail one at a time. The exclusive production of two pions at threshold is very poorly known experimentally, and therefore there are large uncertainties in both the magnitude and the expected distributions of the backgrounds. We rely heavily on the measured single pion distributions, which are outlined in Fig.\,\ref{fig:Primex_sigmas_c1} and tabulated in Table\,\ref{tab:Primex_sigmas}. The major background comes from the $f0(500)$ $0^{+}$ meson that decays to two pions. The production mechanism is expected to very similar to single $\pi^0$ $0^{-}$ production, since the final states are the same except for parity. Therefore, we assume the relative background contributions to the single $\pi^0$ reaction will be similar in our experiment. Production inside the nucleus will tend to reduce hadronic backgrounds in the 2$\pi$ case due to absorption, but we take the conservative approach that absorption does not change the picture substantially. 

We have the following categories of backgrounds:
\begin{itemize}
\item Coherent production: In this case, the target remains intact. Generically, one may classify the two-pion production
according to the sketches in Fig.\,\ref{fig:Diagram}. The left-hand diagram represents the exchange of a virtual photon with the nucleus, i.e. the Primakoff
mechanism. This mechanism is very long range, approximately 100 fm, and is affected minimally by the effects of shadowing or absorption.  This is the signal for the
experiment and our goal is to determine its cross section.
The right-hand diagram represents the exchange of a strongly interacting particle (or trajectory) and effectively results in the production of pions at the
surface of the nucleus. We note that for the neutral pion production, pion exchange is not allowed due to charge conjugation conservation, while 
for the charged pion case, single pion exchange is related to the axial anomaly ($\gamma \pi^0 \rightarrow \pi^+ \pi^-$).  When the interaction leaves the
nuclear target intact, the reaction is referred to as ``nuclear coherent'' and this is our most serious background. 
\item Incoherent production:  When the interaction produces two pions in the quasi-elastic scattering off a single
nucleon, the scattered target usually fragments into particles that range out in the target and are unobservable experimentally. This reaction occurs at larger $-t$ and is generally
kinematically distinct from the signal. The $\pi^0\pi^0$ momentum relative to the photon polarization plane does differentiate between the Primakoff and incoherent production.
\item Any reaction that may be confused with the signal within the experimental resolution or limited acceptance:
 An example of this type of reaction is Primakoff production of $\eta$ mesons, where the $\eta\rightarrow \pi^0 \pi^0 \pi^0$  is
mis-reconstructed as a two-pion final state. 
\end{itemize}

We note that two important backgrounds for the charged-pion polarizability experiment do not contribute in this experiment:
First, coherent $\rho^0$ photo-production is absent in this
experiment because the $\rho^0$ decay into the $\pi^0\pi^0$ channel is prohibited by I-spin conservation.  Second, $\mu^+\mu^-$ production is also not a factor for the neutral pion case.

\subsection{Nuclear coherent background \label{sec:NCback}}
   
 The largest coherent background is  
from the $f_0(500)(J^{PC}=0^{++})$, also called the $\sigma$ meson, and the $f_0(980)$ photo-production.  The width of the
$f_0(980)$ is fairly narrow and does not contribute directly to the strength near threshold.
The $f_0(500)$ width is much
broader, from threshold to 800 MeV, with significant overlap in the
invariant mass region of interest.  Since the $f_0(500)$ is a scalar
particle with the same spin-parity as the $\gamma \gamma \rightarrow
\pi^0\pi^0$ final state near threshold, the azimuthal distribution of the $\pi^0\pi^0$ momentum relative to the
photon polarization plane does not differentiate between coherent
$f_0(500)$ production and the Primakoff reaction.  
This is similar to the Primex-$\pi^0$ experiment, where the dominant background was
nuclear coherent $\pi^0$ photo-production.  The approach used in the
Primex analysis was to measure the $\pi^0$ angular distribution,
effectively the $t$-distribution, then use theoretical calculations of
the angular distributions to separate out contributions from Primakoff
and nuclear coherent. The analysis of the $\pi^0\pi^0$ (NPP) reaction
will approximately parallel what was done for the Primex-$\pi^0$
analysis.  

We parameterize the $\sigma$ meson as detailed in Appendix\,\ref{sec:NCsigma} and assume that the production amplitude can be factorized as
\begin{eqnarray}
\mathcal{A} & = & \mathcal{A}_t(t) \, \mathcal{A}_W(m_{\pi\pi}) \, \mathcal{A}_\tau(\Phi, \phi, \theta),
\end{eqnarray}
where the last factor represents the angular distribution that results in a 
dependence on the di-pion azimuthal angle, $\phi_{\pi\pi}$, of the form $\mathcal{A}_\tau \propto (1 + \mathcal{P} \cos{2\phi_{\pi\pi}}$). 
The mass dependence is given by the S-wave phase shifts that dominate the mass region below 0.8 GeV. We use the approximate description given in 
Appendix\,\ref{sec:ParmSwave}.\footnote{More detailed studies may require including contributions from the D-wave and S-wave, I=2, amplitudes.} 

 \begin{figure}[tbh]
\begin{center}
%\includegraphics[height=5cm,viewport=250 180 0 360,clip=true]{CPP_production3}
\includegraphics[height=5cm,clip=true]{figures/fit_Primakoff_sigma_c1.png}
\includegraphics[height=5cm,clip=true]{figures/PRC80_2009_Fig6.png}
\caption{ Left) Approximation to strong form factor for lead, Right) Figure 6 from Ref.\,\cite{Gevorkyan:2009ge} showing the calculated strong form factor for single $\pi^0$ production off a lead target.
\label{fig:strongFF}}
\end{center}
\end{figure}

We assume the $-t$ dependence of the $\sigma$ has a similar form as for single $\pi^0$ production, namely $\mathcal{A}_t(t) \propto \sin{\theta_{\pi\pi}} \times F_{st}(t)$.  The $\sin{\theta_{\pi\pi}}$ 
comes from the spin-flip required at forward angles to produce a $0^+$ system from a spin-zero target. The factor $F_{st}(t)$ is the strong form factor for the target, which is approximated to
match calculations for the single $\pi^0$ production (Fig.\,6 from Ref.\cite{Gevorkyan:2009ge}). Our Gaussian approximation to the form factor is shown in Fig.\,\ref{fig:strongFF} along side the calculation for single 
$\pi^0$ production. Efforts are underway to calculate the strong form factor for this reaction.\footnote{S. Gevorkyan, private communication.}.
The Primex data showed that the nuclear coherent process is highly
suppressed for heavy nuclei, as shown in Fig.\ref{fig:leaddndt}.  The reason for the suppression is
$\pi^0$ absorption in the nuclear interior, making the coherent
production primarily a surface effect, i.e. proportional to $A$ and
not $A^2$.  For NPP it is expected that suppression of the nuclear  
coherent will be approximately twice stronger than that seen in Primex because two pions
are produced in NPP as compared to a single $\pi^0$ in Primex. 

 \begin{figure}[tbp]
\begin{center}
%\includegraphics[height=5cm,viewport=250 180 0 360,clip=true]{CPP_production3}
\includegraphics[height=15cm,clip=true]{figures/twopi_primakoff_DSelect_test_File_100000_decomposition_PrimNC.pdf}
\caption{Kinematic distributions for the Primakoff signal and nuclear coherent background.
Top left) Two-pi mass, Top right) Two-pi scattering angle, Bottom left) Two-pi azimuthal angle, 
Bottom right) Polar angle of one pion in the $2\pi$ center-of-mass.
\label{fig:decomposition_PrimNC}}
\end{center} 
\end{figure}


Figure\,\ref{fig:decomposition_PrimNC} shows distributions of interest for a sample of Primakoff signal and
nuclear coherent background events in an approximate proportion observed for single pion production. The strong phase between the two production mechanisms is set to
75 degrees for this simulation, where an angle of zero produces maximum interference. This angle must be determined experimentally. 
The distinguishing distributions are shown in the top panel: the 2$\pi$ mass and the 2$\pi$ scattering angle distributions. The Primakoff signal peaks at threshold and at about 0.2 degrees, whereas the nuclear coherent signal rises from threshold as expected from $f0(500)$ production and peaks at an angle of about 0.75 degrees. The azimuthal angular distribution is the same for both signal and background and has no discriminating power. The angular distributions of the pions in
the center of mass of the $2\pi$ system are all uniform, so do not help in distinguishing the signal from the nuclear coherent.  

\begin{figure}[tbp]
\begin{center}
%\includegraphics[height=5cm,viewport=250 180 0 360,clip=true]{CPP_production3}
\includegraphics[width=16cm,clip=true]{figures/twopi_primakoff_DSelect_test_File_20000_IC.pdf}
\caption{Distributions for the incoherent production off free nucleons. The blue crosses represent the generated distributions and the black crosses represent the accepted and kinematically fit distributions.
Left) Two-pi mass. Center) Two-pi scattering angle. The drop of the generated distribution at 1.5 degrees is due to an analysis cut. Right) Two-pi azimuthal angle.
\label{fig:IC}}
\end{center} 
\end{figure}

\subsection{Incoherent two-pion production}
In addition to the coherent production of two pions off the nucleus, two pions may also be produced via the elementary reaction $\gamma N\to \pi^0 \pi^0 N$, breaking up the nucleus in the process. We model the incoherent background with a mass distribution given by the $f0(500)$, but with an exponential $t$ dependence given by $e^{Bt}$, with $B=3.6$ GeV$^{-2}$. The slope is taken from Ref.\cite{Battaglieri:2009aa} and has very large uncertainties. However, as long the slope is small compared to Primakoff production, which has an effective slope of $B\sim560$ GeV$^{-2}$, it does not change the picture. The mass and angular dependencies are shown in Fig.\,\ref{fig:IC}. The strength is small at threshold and at small angles, where Primakoff is strongest. The azimuthal angle is flat, so the photon polarization becomes an important tool in discriminating against this background. 

The cross section for this reaction on free protons is relatively large, about 140 nb/nucleon  for $0.3 < M_{\pi\pi} < 0.8$ GeV.\footnote{The cross section is estimated from the S-wave production of the $f0(500)$ meson extrapolated to small $-t$ from data archived in the {\em hepdata.net} database and reported in Ref.\,\cite{Battaglieri:2009aa}. See Appendix\,\ref{sec:NCsigma} for more information. A factor of one half is applied to the measured cross section for $\pi^+\pi^-$.} However, this process is strongly suppressed in nuclei by Pauli blocking and by pion absorption. The Pauli suppression is proportional to $1-G(t)$, where $G(t)$ is a nuclear form factor and has the limit of $G(t)\to1$ as $-t\to0$ \cite{Gevorkyan:2009ge,primex_inc}. In the case of single $\pi^0$ production, incoherent scattering contributes at the level of a couple of percent, and we expect it to be suppressed more strongly in $2\pi$ production. See Appendix\,\ref{sec:SigmaScaling} for details. Therefore, we expect this background to be about six times smaller than what is used in the present signal extraction studies.

\begin{figure}[tbp]
\begin{center}
%\includegraphics[height=5cm,viewport=250 180 0 360,clip=true]{CPP_production3}
\includegraphics[width=16cm,clip=true]{figures/twopi_primakoff_DSelect_test_File_20000_eta.pdf}
\caption{Kinematically fit distributions for the $\eta$'s produced using the {\em genEtaRegge} event generator.
Left) Two-pi mass. Center) Two-pi scattering angle. Right) Two-pi azimuthal angle.
\label{fig:eta}}
\end{center} 
\end{figure}



\subsection{Miss-identified backgrounds}
There may be important backgrounds that are mistaken for the signal due to miss-identification. These may include
\begin{enumerate}[label=(\roman*)]
    \item coherent production of $\eta$ followed by $\eta\rightarrow \pi^0\pi^0\pi^0 \rightarrow \gamma\gamma\gamma\gamma(\gamma\gamma)$, where only four photons are reconstructed.
    \item production of nucleon resonances that contribute to the $\gamma N \rightarrow N \pi^0\pi^0$ final state. This contribution is expected to be small based on the experience of other Primakoff experiments.
\end{enumerate}
The first reaction is an inelastic, coherent process, and as such
could produce a significant rate for a heavy nuclear target. The kinematics of $\eta$ production was investigated using the {\em genEtaRegge} event generator \cite{hdnote2437}. This event generator has a fairly realistic description of $\eta$ production on hydrogen, including a contribution from the Primakoff reaction. However, it is missing the enhancement provided from scattering off a heavy target. The $\eta$'s were generated with the standard event-generator parameters and were decayed according to their nominal branching fractions. The events were then processed as for the Primakoff signal through the GEANT4 MC and {\em mcsmear}. These steps were followed through the event filter that analyzed the events assuming they were signal, i.e.  $\gamma Pb \rightarrow Pb\, \pi^0 \pi^0$ and a kinematic fit was performed assuming the target was missing. Standard missing-mass and $\chi^2$ cuts were used to pick out events that mimicked the signal. The distributions of the accepted events are shown in Fig.\ref{fig:eta}. 
We have scaled the event rate for $\eta$ production on lead using the scaling rules described in Appendix\,\ref{sec:SigmaScaling}, resulting in ten times the rate for Primakoff production of two pions ($\sim \Gamma_\eta\times Br(\eta \rightarrow \pi^0)/\Gamma_\sigma$).

We use a couple of very simple but powerful selection cuts to remove the $\eta$ background. The cuts include a selection on the missing mass squared ($|MM - M_{Pb}{^2}| < 0.1$ GeV$^2$) and a cut on the $\chi^2 < 5$ of the kinematic fit to $\gamma\,Pb\rightarrow \pi^0\pi^0 (Pb)$ with a missing recoil. In the analysis we also require a $2\pi$ scattering angle  ($\theta_{\pi\pi} <1.5$ degrees), which further restricts the range of missing mass. The result of these selections, which have been applied uniformly to signal and background, reduce the contamination from this source to about 16\% of the signal. These selections are illustrative and will allow us to achieve our experimental goals but further optimizations are likely. We note that the absolute number of mis-identified $\eta$'s in our signal region can be determined empirically from the measured rate of fully reconstructed $\eta\rightarrow\pi^0\pi^0\pi^0$ events.

 \begin{figure}[tbp]
\begin{center}
%\includegraphics[height=5cm,viewport=250 180 0 360,clip=true]{CPP_production3}
\includegraphics[height=15cm,clip=true]{figures/twopi_primakoff_DSelect_test_File_100000_decomposition_PrimNCICeta.pdf}
\caption{Kinematic distributions for the Primakoff signal and nuclear coherent background.
Top left) Two-pi mass, Top right) Two-pi scattering angle, Bottom left) Two-pi azimuthal angle, 
Bottom right) Polar angle of one pion in the $2\pi$ center-of-mass.
\label{fig:decomposition_PrimNCICeta}}
\end{center} 
\end{figure}

\subsection{Extraction of the Primakoff signal \label{sec:signalfit}}
The Primakoff signal is determined using an amplitude fit\footnote{AmpTools, https://github.com/mashephe/AmpTools/wiki.} to all data simultaneously. It assumes that the mass and angular distributions are know for each of the contributions to the 2$\pi$ sample. A complex scaling factor is determined for each contribution by doing an unbinned maximum likelihood fit to the event sample. The result of such a fit to the sample that includes the Primakoff and nuclear coherent processes only is shown in  Fig.\,\ref{fig:decomposition_PrimNC}. The fit to the data sample that also includes both incoherent and broken $\eta$'s is shown in Fig.\,\ref{fig:decomposition_PrimNCICeta}. The three kinematic quantities that are most discriminating between the signal and background are the 2$\pi$ mass $M_{\pi\pi}$, the 2$\pi$ scattering angle $\theta_{\pi\pi}$ and the azimuthal angle $\phi_{\pi\pi}$. The fit uses all the kinematic information contained in the event sample to determine the signal and background components that are present. As demonstrated in the figure, a good fit is obtained to the data and Primakoff signal is determined. The statistical uncertainties as a function of $M_{\pi\pi}$ are obtained directly from the fit (Top left plot in Fig.\,\ref{fig:decomposition_PrimNCICeta}). We estimate the systematic uncertainty (3\%) in the extraction of the signal by varying the fraction of incoherent in the sample. 


\section{Photon beam flux  \label{sec:flux}   }

\subsection{Photon beam flux accounting with the GlueX pair spectrometer}
The photon beam flux can be directly extracted by analyzing the pair
spectrometer (PS) data with the thin beryllium converter installed in
the beam in from of it.  The absolute normalization of the PS
performed with the total absorption counter (TAC) during the dedicated
run.

The systematics from the photon beam flux accounting by pair
spectrometer is originated from few main contributions: overall
spectrometer calibration with TAC quality; accuracy of the Monte-Carlo
simulation of this process; long term stability of the spectrometer
performance; and change of conditions between low intensity beam (TAC
calibration) and production intensity.  There are few other less
significant contributions. GlueX PS acceptance \cite{hdnote3684} shown
on Fig.~\ref{fig:psacc}. For the proposed experiment PS magnetic field
should be reduced to cover the beam energy range $5-6\,GeV$.
\begin{figure}[tpb]
\begin{center}
\includegraphics[width=10cm,angle=0]{figures/ps_acceptance.pdf}
\end{center}
\caption{GlueX PS acceptance extracted from TAC data analysis (blue points);
red points -- Monte-Carlo simulation}
\label{fig:psacc}
\end{figure}
The methodology and accuracy of the PS analysis is the same as in
PrimEx-D experiment, currently running in Hall-D, and has value
$\sim1-1.5\,\%$ \cite{PrimexDexp}.

\subsection{Cross section verification with the exclusive single $\pi^{0}$ photoproduction  \label{sec:pi0norm} }
The extracted cross section can also be normalized on or independently
from PS analysis verified with the $\pi^0$ radiative decay width
extraction.  Fig.~\ref{fig:leaddndt} shows exclusive single $\pi^0$
photoproduction yield at forward angle obtained by the PrimEx
experiment and used for $\pi^0$ radiative decay width extraction.

The photon beam flux in PrimEx was $0.725\times10^{12}$ for
4.9-5.5$\,$GeV bremsstrahlung spectrum part on 5\% rad. len. lead
target. The distance between calorimeter and target was $\sim7.3\,m$
and the central square part of the calorimeter, used in analysis was
$\sim70\times70\,cm$. These conditions have to be compared with the
proposed experiment conditions: 20 days of $~10^7$ collimated beam
photon/sec (i.e. 20 times more than PrimEx lead target beam flux), the
distance between target and FCAL $\sim6.2\,m$ and active calorimeter
part diameter $\sim2\,m$.  The central hole with one calorimeter
modules layer around which should be excluded from the analysis for
PrimEx case was $\sim8\times8\,cm$ and for FCAL $\sim20\times20\,cm$,
which is decreasing FCAL acceptance at forward angle. Comparison of
these experimental conditions allows us expecting an order of
magnitude higher exclusive single $\pi^0$ photoproduction statistics.
Thus PrimEx statistical uncertainty for lead will be decreased from
$\sim2.5\,\%$ down to $\sim1\,\%$.  For the systematical uncertainty,
in PrimEx it was $\sim2.1\,\%$ and has two major contributions: yield
extraction ($\sim1.6\,\%$) and photon beam flux accounting
($\sim1.0\,\%$).  The first contribution is partly statistically
driven and reduces with increasing of statistics; and the second one
cancels out since it is the same photon beam flux for the single and
double exclusive $\pi^0$ photoproduction.  The main factors increasing
systematics for the proposed experiment are: the angular resolution of
FCAL is about a factor of two worse than for PWO crystals used in the
PrimEx analysis;
%the single $\pi^0$ photoproduction theory needs to be involved since
%the photon beam energy spectra are not the same for PrimEx and
%proposed experiment;
and the magnetic field is not swiping out charged background like it
was in PrimEx.  As a result we can expect slightly worse systematical
uncertainty than in PrimEx and statistical precision of $\sim1\,\%$,
i.e. total error $2.5-3.5\,\%$ for $\pi^0$ radiative width extraction
(excluding absolute photon beam flux accounting, target number of
atoms and partly FCAL trigger efficiency contributions to the
systematics which are canceling out).  The expected total beam flux
uncertainty for such a normalization should also include the PrimEx
total error of the $\pi^0$ radiative width, which was recently
reported as 1.5\% \cite{Larin:2018}.  All this gives $\sim3-4\,\%$
error for photon beam flux from normalization to the re-extracted $\pi^0$ radiative decay width.

\subsection{Muon pair production}
In addition to these normalization channels, production of muon pairs,
which has a known cross section, can be used as a measurement of
photon flux. Since the experiment will be running concurrently with
the Charged Pion Polarizability (CPP) experiment, the photon flux on
target will be the same by definition. CPP plans to use muon pair
creation by beam photons as its main normalization channel, and so
those measurements will be available for normalization of the neutral
pion channel as well. In the case of CPP, the GlueX track finding and
fitting efficiency will have to be determined for muon pairs, but any
systematic error in that determination will largely cancel when
applied to charged pion pairs. That will not be the case for the
neutral pion channel and will have to be taken into account when
evaluating systematic errors due to this method of normalization. In
any case, muon pair production should provide a useful check on the
other methods mentioned above.

\section{Errors and Sensitivity}
We summarize the anticipated errors in the determination of the $\pi^0$ polarizability. We assume 
20 days of running on a 5\% radiation length $^{208}$Pb target, 10$^7$ photons/s, and nominal acceptance for $\pi^0 \pi^0$.
Table \ref{errors} summarizes the estimated statistical and systematic errors. In the following we describe each of
these contributions in detail: 

%\end{landscape}
% \begin{landscape}
\begin{table}[bt]
\caption{Uncertainties in the extraction of $\pi^0$ polarizabilities $\alpha_{\pi^0}-\beta_{\pi^0}$.
\label{errors}
}
\begin{center}
\begin{tabular}{|l|l|c|c|}
\hline
\hline
  &  Source  & Uncertainty   \\  \hline \hline
  1 & Statistical uncertainty  &  2.3 \%    \\ \hline
%  2 & Flux normalization &  1.5 \%     \\ \hline
  2 & Signal extraction  & 3.0 \%  \\ \hline
  3 & Detector acceptance and efficiency &  3.5 \%   \\ \hline
  4 & Total systematic error  &  4.6\% \\ \hline
  5 & Total error on cross section  &  5.1\% \\ \hline
  6 & Projected error in $\alpha - \beta$ &  39\%  \\ 
 \hline
 \hline
\end{tabular}
\end{center}
\end{table}
%\end{landscape}

\begin{enumerate}

\item
Statistical uncertainty in extraction of the Primakoff signal as determined by the fit shown in the top left plot in Fig.\,\ref{fig:decomposition_PrimNCICeta} (Section\,\ref{sec:signalfit}).

% \item Flux normalization. We have several methods for determining the flux (Section\,\ref{sec:flux}). The Primex-D experiment expects an uncertainty of 1.5\% and we use that as our estimate here.

\item
Systematic uncertainty in extraction of the signal. This contribution is estimated to be 3\% based on differences obtained in the Primakoff signal by varying the amount of background contributions (Section\,\ref{sec:signalfit}).

\item 
Detector acceptance and efficiency. We measure the cross section for two-pion production relative to single-pion production (Section\,\ref{sec:pi0norm}). The flux factors cancel. Remaining is the uncertainty in the detector acceptance times efficiency relative to $\gamma \mathrm{Pb} \rightarrow \pi^0 \mathrm{Pb}$, which we estimate to be 3.5\%.

\item
Total systematic error (items 2-3): combining the systematic errors in quadrature gives 4.6\%.

\item 
Error on cross section (quadrature sum of items 1 and 4): 5.1\%.

\item
The current estimate by Dai and Pennington (Table II in Ref.\,\cite{Dai:2016ytz}) indicates
that a 13\% determination of
$\sigma(\gamma\gamma\rightarrow\pi^0\pi^0)$ will determine the
combination $\alpha_{\pi^0}-\beta_{\pi^0}$ to a precision of 100\%, i.e., $\Delta(\alpha_{\pi^0}-\beta_{\pi^0}) \sim 7.7\Delta(\sigma$) .
From here we estimate that our uncertainty on $\Delta(\alpha_{\pi^0}-\beta_{\pi^0}) \sim$ 39\%. We note that the basis for extracting the polarizabilities may be improved
in the near future and theoretical effort is being directed specifically toward this goal.

\end{enumerate}

\begin{table}[hbt]
\caption{Approved beam request and running conditions for CPP. NPP would run concurrently.
\label{request}
}
\begin{center}
\begin{tabular}{|l|c|c|c|c|c|c|c|c|}
\hline
\hline
  Running condition  &            \\ \hline
  Days for production running  &   20   \\ \hline
  Days for calibrations &  5       \\ \hline
  Target   & $^{208}$Pb   \\ \hline
  Photon intensity in coherent peak &   10$^7$ photons/s     \\ \hline
  Edge of coherent peak  &  6 GeV   \\ \hline
 \hline
 \hline
\end{tabular}
\end{center}
\end{table}
 
\section{Summary and beam request}
We have investigated the possibility of determining the neutral pion
polarizabilities $\alpha_{\pi^0}-\beta_{\pi^0}$, a quantity for which there are no existing measurements.
Our proposal is to extract the polarizability from a 
measurement of the cross section of the Primakoff reaction $\gamma
\rm{Pb}\rightarrow \pi^0 \pi^0 \rm{Pb}$. We propose to make this
measurement using data taken simultaneously with the CPP\cite{CPPexp}
experiment in Hall D. Table \ref{request} summarizes the approved beam request for the CPP experiment.
The existing GlueX detector has sufficient
resolution and high acceptance for this process. We expect to collect approximately 1800 signal events during the
approved 20 PAC days. The anticipated   uncertainties on
the signal represent a significant improvement over existing data as shown in Fig.\,\ref{fig:sigma_2pi0_figs_4}.
Using the estimate by Dai and Pennington \cite{Dai:2016ytz} we expect to be able to make the first extraction of the 
$\alpha_{\pi^0}-\beta_{\pi^0}$ polarizability with an uncertainty of 39\%.

\begin{figure}[tpb]
\centering
\includegraphics[page=4,width=4.75in]{figures/sigma_2pi0_figs.pdf}
\caption{Estimated uncertainties (systematic and statistical errors are added in quadrature) in determining $\sigma(\gamma\gamma\rightarrow\pi^0\pi^0$) during 20 PAC days running simultaneously with the approved CPP experiment. The data points from the single previous Cristal Ball measurement \cite{Marsiske:1990hx} are shown for comparison.
\label{fig:sigma_2pi0_figs_4}}
\end{figure}






%\input{JLAB_MuonDetector.tex}



%---------------------------------------------------------------------------------------------------------------
%--------------------------------- References-----------------------------------------------------------------
%---------------------------------------------------------------------------------------------------------------
\clearpage

%\nocite{*}
\bibliographystyle{unsrt}                                                                              
\bibliography{Pi0Polarizability}   


\end{document}
